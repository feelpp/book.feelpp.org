\chapter{Approximation de problèmes coercifs}
\label{cha:appr-de-probl}

Dans ce chapitre, on s'intéresse à l'approximation de problèmes coercifs:
\begin{itemize}
\item le Laplacien et des variantes sur les conditions aux limites et le
  l'équation elle-même;
\item l'élasticité linéaire qui permet de modéliser de petites déformations
  d'un milieu continu déformable.
\end{itemize}

\section{Le Laplacien}
\label{sec:le-laplacien}

on s'instéresse dans cette section à l'\emph{approximation élément fini
conforme} du problème suivant:
\begin{problem}
  \label{prob:1}
  On cherche $u$ telle que
  \begin{equation}
    \label{eq:64}
    \begin{split}
      -\Delta u &= f \mbox{ dans } \Omega\\
      u &= 0 \mbox{ sur } \partial \Omega
    \end{split}
  \end{equation}
\end{problem}


\subsection{Cadre Mathematique}
\label{sec:cadre-mathematique}

On suppose que $f \in L^2(\Omega)$. La formulation faible du
problème~\ref{prob:1} est la suivante:
\begin{problem}
  \label{prob:2}
  On cherche $u \in H^1_0(\Omega)$ telle que
  \begin{equation}
    \label{eq:65}
    \ds\int_\Omega \nabla u \cdot \nabla v = \ds \int_\Omega f\, v,\quad \forall v \in H^1_0(\Omega)
  \end{equation}
\end{problem}

\subsubsection{Problème bien posé}
\label{sec:probleme-bien-pose}

Introduisons
\begin{itemize}
\item $V = H^1_0(\Omega)$ doté de la norme $\|\cdot\|_{1,\Omega}$ telle que
  $\|v\|_{1,\Omega} = (\|v\|^2_{0,\Omega} + \|\nabla v\|^2_{0,\Omega})^{1/2}$
\item la forme bilinéaire $a \in \mathcal{L}(V \times V, \R)$ telle que $a(u,v) = \int_\Omega
  \nabla u \cdot \nabla v $
\item la forme linéaire $\ell \in \mathcal{L}(V, \R)$ telle que $l(v) = \int_\Omega
  f \nabla v $
\end{itemize}

Le problème~\ref{prob:2} s'écrit sous forme abstraite
\begin{problem}
  \label{prob:5}
  On cherche $u \in V$ telle que
  \begin{equation}
    \label{eq:65}
    a(u, v) = \ell(v), \quad \forall v \in V
  \end{equation}
\end{problem}

L'espace $V$ est un espace de Hilbert et les formes $a$ et $\ell$ sont
continues sur $V\times V$ et $V$ respectivement. Il ne reste plus qu'à
vérifier si le problème est bien posé (existence d'une solution unique). Pour
cela on utilise démontre la \emph{coercivité} de la forme bilinéaire $a$ sur
$V \equiv H^1_0(\Omega)$. Ceci se fait grâce au lemme suivant:
\begin{lemma}[Inégalité de Poincaré]
  \label{lem:1}
  Soit $\Omega$ un ouvert borné de $\R$. Il existe une constante $c_\Omega$
  (dépendente de $\Omega$ telle que
  \begin{equation}
    \label{eq:66}
    \forall v \in H^1_0(\Omega),\quad \|v\|_{0,\Omega} \le c_\Omega \|\nabla v\|_{0,\Omega}
  \end{equation}
\end{lemma}
\begin{remark}
  \label{rem:24}
  $c_\Omega$ est homogène à une longeur et peut être interprétée comme une
  mesure caractéristique de $\Omega$
\end{remark}

Grâce à l'inégalité de Poincaré, on a le résultat suivant
\begin{proposition}
  \label{prop:7}
  La forme bilinéaire $a$ du problème~\ref{prob:2} est \emph{coercive}

  \begin{proof}
    On note tout d'abord que par l'inégalité de Poincaré et la définition de
    $\|\cdot\|_{1,\Omega}$
    \begin{equation}
      \label{eq:68}
      \|v\|^2_{1,\Omega} \le (1 + c^2_\Omega) \|\nabla v\|^2_{0,\Omega}
    \end{equation}
    On en déduit que
    \begin{equation}
      \label{eq:67}
      \forall v \in H^1_0(\Omega),\quad a(v,v) = \|\nabla v\|^2_{0,\Omega} \ge
      \ds \frac{1}{1+c^2_\Omega} \|v\|^2_{1,\Omega}
    \end{equation}
    Le Lemme de Lax-Milgram~\ref{thr:12} permet alors de conclure sur l'existence d'une
    solution unique pour le problème~\ref{prob:2}.
  \end{proof}
\end{proposition}

\subsubsection{Approximation conforme}
\label{sec:appr-conf}

On utilise une approximation conforme par éléments finis de Lagrange. On
considère $\Omega$ un polygone ou polyhèdre régulier de $\R[2]$ ou $\R[3]$
respectivement et un maillage $\calTh = \{K_e\}_{e=1...\Ne}$ de $\Omega$. On considère un élément
fini de référence $(\hat{K},\hat{P},\hat{\Sigma})$ tel que $\Pk \subset \hat{P}$
et $k+1 > \frac{d}{2}$, voir le théorème~\ref{thr:16}. On note
\begin{equation}
  \label{eq:70}
  L^k_{c,h} = \{ v_h \in C^0(\bar{\Omega}); \forall K \in \mathcal{T}_h,\ v_h
  \circ T_K \in \hat{P}\}
\end{equation}
où $T_K$ est la tranformation géométrique de $\hat{K}$ dans $K$.
\begin{itemize}
\item Si on utilise $\hat{P}=\Pk$ on a $L^k_{c,h} = P^k_{c,h}$.
\item Si on utilise $\hat{P}=\Qk$ on a $L^k_{c,h} = Q^k_{c,h}$.
\end{itemize}
Afin de construire un espace d'approximation conforme \ie $V_h \subset V
=H^1_0(\Omega)$ on prend
\begin{equation}
  \label{eq:71}
  V_h = L^k_{c,h} \cap H^1_0(\Omega)
\end{equation}
c'est à dire que les fonctions de $V_h$ satisfont les conditions aux limites
outre le fait d'être dans $L^k_{c,h}$. Le problème discret s'écrit alors
\begin{problem}
  \label{prob:6}
  Trouver $u_h \in V_h$ telle que
  \begin{equation}
    \label{eq:72}
    a(u_h,v_h) = \ell(v_h),\, \forall v_h \in V_h
  \end{equation}
\end{problem}
qui est bien posé (existence et unicité de $u_h$) car $a$ est coercive sur $V$
et que $V_h \subset V$. On a le résultat suivant:
\begin{theorem}[Convergence de $u_h$]
  \label{thr:17}
  On suppose que $u$ solution de \ref{prob:2} est dans $H^{k+1}(\Omega) \cap
  H^1_0(\Omega)$, alors il existe une constante $c_1$ telle que pour tout $h$
  \begin{equation}
    \label{eq:73}
    \|u-u_v\|_{1,\Omega} \le c_1 h^k |u|_{k+1,\Omega}
  \end{equation}
  et il existe une constante $c_2$ telle que pour tout $h$
  \begin{equation}
    \label{eq:74}
    \|u-u_v\|_{0,\Omega} \le c_2 h^{k+1} |u|_{k+1,\Omega}
  \end{equation}
\begin{proof}
  La preuve de (\ref{eq:73}) est obtenue grâce au lemme de Cea~(\ref{eq:cea})
  et du théorème d'interpolation~(\ref{eq:62}).
  La preuve de (\ref{eq:74}) est obtenue grâce au lemme d'Aubin-Nitsche qui
  permet d'affirmer qu'il existe une constante $c$ telle que
  \begin{equation}
    \label{eq:75}
    \|u-u_h\|_{0,\Omega} \leq c h |u-u_h|_{1,\Omega}
  \end{equation}
  et donc que (\ref{eq:74}) se déduit de (\ref{eq:74}).
\end{proof}
\end{theorem}

\subsubsection{Implémentation avec \Feel}
\label{sec:impl-en-feel++}

Avec \Feel, on ne construit pas explicitement l'espace $V_h$ mais
$L^k_{c,h}$. Le traitement des conditions aux limites de Dirichlet du problème
(\ref{eq:64}) peut être effectué de diverses fa\c {c}ons, nous en verrons une.

Commencons par le maillage, dans un premier temps nous définissons le type du
maillage contenant soit des éléments de type simplexe~(segment, triangle,
tetrahèdre) ou de type hypercube~(segment, quadrangle, hexahèdre).

\begin{lstlisting}[caption={Maillage $\calTh$}]
  // un maillage de simplexe dans $\R$ telle que la transformation
  // géométrique $T_K,\ K \in \calTh$  soit affine
  typedef Mesh<Simplex<d,1> > mesh_type;

  // un maillage d'hypercube dans $\R$ telle que la transformation
  // géométrique $T_K,\ K \in \calTh$  soit affine en chacune des variables
  // typedef Mesh<Hypercube<d,1> > mesh\_type;

  // generate the mesh associated to the unit square $[0,1]^2$ using triangles
  auto mesh = unitSquare();
\end{lstlisting}
\begin{remark}
  \label{rem:25}
  Le mot clé \texttt{auto} permet de faire de l'inférence de type, pour plus
  de détails consultez
  \href{http://fr.wikipedia.org/wiki/C%2B%2B11#Inf.C3.A9rence_de_types}{la page C++11 de Wikipedia}.
\end{remark}
Ensuite nous pouvons définir l'espace $L^k_{c,h}$,
\begin{lstlisting}[caption={$L^k_{c,h}$}]
  // Vh est une structure de donnée allouée dynamiquement
  auto Vh = Pch<1>( mesh );
  // u est un élément de Vh
  auto u = Vh->element();
  // u est un autre élément de Vh
  auto u = Vh->element();
\end{lstlisting}

À présent, nous définissons les formes bilinéaires $a$ et $\ell$ qui sont
respectivement des formes bilinéaires et linéaires.

\begin{lstlisting}[caption={Définitions de $a$ et $\ell$}]
  // $a \in \mathcal{L}(V_h \times V_h,\ \R[\phantom{1}])$
  auto a = form2( _test=Vh, _trial=Vh );
  // $a = \sum_{e=1...\Ne} \int_{K_e} \nabla \varphi_j \cdot \nabla \varphi_i,\quad  i,j=1...,\dim{V_h}$
  a = integrate( _range=elements(mesh),
                 _expr=gradt(u)*trans(grad(v)) );

  // $\ell \in \mathcal{L}(V_h,\ \R[\phantom{1}])$
  auto l = form1( _test=Vh );
  // $\ell = \sum_{e=1...\Ne} \int_{K_e} f  \varphi_i,\quad  i=1...,\dim{V_h}$
  l = integrate( _range=elements(mesh), _expr=f*id(v) );
\end{lstlisting}

Afin de traiter les conditions aux limites de Dirichlet homogènes, on peut
utiliser le mot-clé \texttt{on} qui permet de les imposer de manière forte.
\begin{lstlisting}[caption={Traitement des conditions de Dirichlet avec \texttt{on}}]
  a += on(_range=boundaryfaces(mesh), _element=u, _rhs=l, _expr=constant(0.) );
\end{lstlisting}
\begin{remark}
  \label{rem:26}
  Le mot-clé \texttt{constant} permet de transformer une type numérique (\eg
  entier, flottant) en une expression utilisable par le langage de
  \Feel. Notez également l'opération \texttt{+=} qui permet de rajouter le
  traitement des conditions de Dirichlet tout en gardant les contributions
  précédentes. L'opération \texttt{=} aurait d'abord remis à $0$ les entrées
  de la matrice associée à $a$.
\end{remark}

Enfin nous pouvons résoudre le problème~\ref{prob:6}

\begin{lstlisting}[caption={Résolution du problème discret~\ref{prob:6}}]
  a.solve( _rhs=l, _solution=u );
\end{lstlisting}

Le listing complet
\lstinputlisting[linerange=12-25,caption={Exemple Laplacien avec conditions de
Dirichlet homogènes}]{../Codes/prudhomme/laplacian/dirichlet_homogene.cpp}


\subsection{Conditions aux limites}
\label{sec:cond-aux-limit}

\subsubsection{Conditions aux limites de Dirichlet non homogène}
\label{sec:cond-aux-limit-1}

On suppose toujours $f \in L^2(\Omega)$ et on se donne une fonction $g \in
C^{0,1}(\partial \Omega)$\footnote{$g$ est Lipschitzienne sur $\partial
\Omega$}. On s'intéresse au problème suivant:
\begin{problem}
  \label{prob:7}
  On cherche $u : \Omega \rightarrow \R[]$ telle que
  \begin{equation}
    \label{eq:76}
    \begin{split}
    -\Delta u &= f \mbox{ dans } \Omega\\
    u &= g \mbox{ sur } \partial \Omega
    \end{split}
  \end{equation}
\end{problem}

\begin{remark}
  \label{rem:27}
  L'hypothèse $g \in C^{0,1}(\partial \Omega)$ permet d'affirmer qu'il existe
  $u_g \in H^1(\Omega)$ telle que $u_{g_{|\partial \Omega}} = g$.
\end{remark}

On se ramène au problème avec conditions de Dirichlet homogène en faisant le
change d'inconnue $u_0=u-u_g$ et on s'intéresse au problème suivant:
\begin{problem}
  \label{prob:8}
  On cherche $u_0 \in H^1_0(\Omega)$ telle que
  \begin{equation}
    \label{eq:77}
    a(u_0,v) = \ell(v) - a(u_g,v),\quad \forall v \in H^1_0(\Omega)
  \end{equation}
\end{problem}
Ce problème est bien posé d'après Lax-Milgram, voir section précédente.

\begin{theorem}
  \label{thr:18}
  On suppose que $u$ solution de \ref{prob:8} est dans $H^{k+1}(\Omega) \cap
  H^1_0(\Omega)$, alors il existe une constante $c_1$ telle que pour tout $h$
  \begin{equation}
    \label{eq:73}
    \|u-u_v\|_{1,\Omega} \le c_1 h^k |u|_{k+1,\Omega}
  \end{equation}
  et il existe une constante $c_2$ telle que pour tout $h$
  \begin{equation}
    \label{eq:74}
    \|u-u_v\|_{0,\Omega} \le c_2 h^{k+1} |u|_{k+1,\Omega}
  \end{equation}
\end{theorem}

Avec \Feel, les conditions Dirichlet non-homogènes sont traitées par exemple
avec le mot-clé \texttt{on}.
\begin{lstlisting}
  // définition de la fonction, p.ex $g=sin(2 \pi x)$
  auto g = sin(2*pi*Px() );
  // définition de $a$
  ...
  // ajout des conditions de Dirichlet non-homogène
  a += on( _range=boundaryfaces(mesh), _expr=g );
\end{lstlisting}
\begin{remark}
  \label{rem:28}
  Il n'y a pas besoin de rajouter le terme $a(u_g,v)$ au second membre $\ell(v)$, il est
  pris en compte automatiquement par \texttt{on}.
\end{remark}
Voici le listing complet de l'exemple du laplacien avec conditions de Dirichlet non-homogène
\lstinputlisting[linerange=12-26,caption={Exemple Laplacien avec conditions de
Dirichlet non-homogènes}]{../Codes/prudhomme/laplacian/dirichlet_nonhomogene.cpp}

\subsubsection{Condition aux limites de Neumann}
\label{sec:cond-aux-limit-2}

Étant donnés un réel $\mu$ strictement positif, $f \in L^2(\Omega)$ et $g \in
L^2(\partial \Omega)$, on s'intéresse au problème suivant:
\begin{problem}
  \label{prob:9}
  On cherche $u : \Omega \rightarrow \R[]$ telle que
  \begin{equation}
    \label{eq:78}
    \begin{split}
      -\Delta u + \mu u &= f, \mbox{ dans } \Omega\\
      \partial_\Next u &= g, \mbox{ sur } \partial\Omega
    \end{split}
  \end{equation}
\end{problem}
où $\partial_\Next u = \nabla u \cdot \Next = \sum_{i=1}^d n_i \partial_i u$
dénote la dérivée normale de $u$ avec $\Next=(n_1,...,n_d) \in \R[d]$ la
normale extérieure unitaire en un point du bord de $\Omega$.
La formulation faible s'écrit
\begin{problem}
  \label{prob:13}
  On cherche $u \in H^1(\Omega)$ telle que
  \begin{equation}
    \label{eq:79}
    a( u, v ) = \ell(v),\ \forall v \in H^1(\Omega)
  \end{equation}
  avec
  \begin{equation}
    \label{eq:80}
    a( u, v ) = \ds \int_\Omega \nabla u \cdot \nabla v + \mu u v
  \end{equation}
  et
  \begin{equation}
    \label{eq:81}
    \ell( v ) = \ds \int_\Omega f v + \int_{\partial\Omega} g v
  \end{equation}
\end{problem}

On a
\begin{equation}
  \label{eq:82}
  a(v, v) = \ds \int_\Omega \nabla v \cdot \nabla v + \mu v v \ge \min(1,\mu)
  \int_\Omega \nabla v \cdot \nabla v +  v v  = \min(1,\mu) \|v\|_{1,\Omega}, \quad
  \forall v \in H^1(\Omega)
\end{equation}
ce qui nous permet d'affirmer que $a$ est coercive sur $H^1(\Omega)$ et que le
problème~\ref{prob:13} est bien posé grâce à Lax-Milgram.

\lstinputlisting[linerange=12-25,caption={Exemple Laplacien avec conditions de
Neumann}]{../Codes/prudhomme/laplacian/neumann.cpp}

\subsubsection{Conditions aux limites de Robin}
\label{sec:cond-aux-limit-3}

Étant donnés un réel $\mu$ strictement positif, $f \in L^2(\Omega)$ et $g \in
L^2(\partial \Omega)$, on s'intéresse au problème suivant:
\begin{problem}
  \label{prob:14}
  On cherche $u : \Omega \rightarrow \R[]$ telle que
  \begin{equation}
    \label{eq:83}
    \begin{split}
      -\Delta u  &= f, \mbox{ dans } \Omega\\
      \mu u + \partial_\Next u &= g, \mbox{ sur } \partial\Omega.
    \end{split}
  \end{equation}
\end{problem}
La formulation faible s'écrit
\begin{problem}
  \label{prob:15}
  On cherche $u \in H^1(\Omega)$ telle que
  \begin{equation}
    \label{eq:84}
    a( u, v ) = \ell(v),\ \forall v \in H^1(\Omega)
  \end{equation}
  avec
  \begin{equation}
    \label{eq:85}
    a( u, v ) = \ds \int_\Omega \nabla u \cdot \nabla v + \int_{\partial
    \Omega} \mu u v
  \end{equation}
  et
  \begin{equation}
    \label{eq:86}
    \ell( v ) = \ds \int_\Omega f v + \int_{\partial\Omega} g v
  \end{equation}
\end{problem}

On a
\begin{equation}
  \label{eq:69}
  \begin{split}
    a(v, v) & = \ds \int_\Omega \nabla v \cdot \nabla v + \int_{\partial\Omega} \mu v v \\
    & \geq \min(1,\mu)\left( \int_\Omega \nabla v \cdot \nabla v +
      \int_{\partial\Omega} v v\right)  \\
    &\geq \min(1,\mu) \|v\|_{1,\Omega}, \quad \forall v \in H^1(\Omega)
  \end{split}
\end{equation}
La forme bilinéaire $a$ est donc coercive et le problème~\ref{prob:15} est
bien posé grâce à Lax-Milgram.

On considère le problème discret suivant
\begin{problem}
  \label{prob:11}
  On cherche $u_h \in L^k_{c,h}(\Omega)$ telle que
  \begin{equation}
    \label{eq:87}
    a(u_h,v_h) = \ell(v_h)\quad \forall v_h \in L^k_{c,h}
  \end{equation}
\end{problem}
Le problème est bien posé  ($L^k_{c,h} \subset H^1(\Omega)$). La convergence de
$u_h$ est donnée par le théorème~\ref{thr:17}.

Considérons $\Omega=[0,1]^2$ et les données $\mu=0.01$, $f=1$ et $g=0$. Le
code \Feel permettant de résoudre le problème discret~\ref{prob:11}, voir
\texttt{../Codes/prudhomme/laplacian/robin.cpp} pour le listing complet.
\lstinputlisting[linerange=12-26,caption={Exemple Laplacien avec conditions de
Robin}]{../Codes/prudhomme/laplacian/robin.cpp}


\subsection{Advection-diffusion-réaction avec diffusion dominante}
\label{sec:advection-diffusion}

On s'intéresse au problème suivant:
\begin{problem}
  \label{prob:10}
  On cherche $u : \Omega \rightarrow \R[]$ telle que
  \begin{equation}
    \label{eq:30}
    \begin{split}
      -\nabla \cdot ( {\bm \alpha} \nabla u  ) + {\bm \beta} \cdot \nabla u + \mu u &= f
      \mbox{ dans } \Omega\\
      u &= 0 \mbox{ sur } \partial \Omega\\
    \end{split}
  \end{equation}
\end{problem}

La variations sur les conditions aux limites de la
section~\ref{sec:cond-aux-limit} s'appliquent.
\begin{itemize}
\item $-\nabla \cdot ( {\bm \alpha} \nabla u  )$ est un terme de diffusion,
\item ${\bm \beta} \cdot \nabla u$ est un terme de convection,
\item $\mu u$ est un terme de réaction
\end{itemize}
Ce type d'équation est très fréquente en ingéniérie, biologie ou encore finance.

On suppose que ${\bm \alpha} \in [L^{\infty}(\Omega)]^{d,d}$, ${\bm \beta} \in
[L^{\infty}(\Omega)]^{d}$ et $\mu \in L^\infty(\Omega)$. On suppose que
l'opérateur $\mathcal{L}$ tel que $\mathcal{L} u = -\nabla \cdot ( \alpha \nabla u
) + \beta \cdot \nabla u + \mu u$ est elliptique au sens suivant:

\begin{definition}[Opérateurs Elliptiques]
  \label{def:39}
  L'opérateur $\mathcal{L}$ est elliptique si il existe une constante $\alpha_0$
  telle que presque pour tout $x\in\Omega$
  \begin{equation}
    \label{eq:88}
    \forall \xi=(\xi_1, \ldots , \xi_n)\in\RR^n,\quad {\ds \sum_{i,j=1}^n  \alpha_{ij}(x) \, \xi_i \, \xi_j  \ge \alpha_0 \, \| \xi \|^2 }
  \end{equation}
\end{definition}

\begin{remark}
  \label{rem:29}
  Le Laplacien est dans la catégorie est opérateurs elliptiques, il correspond à
  ${\bm \beta} = 0$, $\mu = 0$ et ${\bm \alpha}=\mathcal{I}_d$ avec $\mathcal{I}_d$ la
  matrice identité de $\R[d,d]$
\end{remark}


La formulation faible s'écrit
\begin{problem}
  \label{prob:12}
  On cherche $u \in H^1_0(\Omega)$ telle que
  \begin{equation}
    \label{eq:89}
    a(u,v) = \ell(v) \quad \forall v \in H^1_0(\Omega)\\
  \end{equation}
  avec
  \begin{equation}
    \label{eq:90}
    a(u,v)=\int_\Omega ({\bm \alpha} \nabla u ) \cdot \nabla v + ({\bm \beta} \cdot
    \nabla u) v + \mu u v
  \end{equation}
  et
  \begin{equation}
    \label{eq:91}
    \ell(v) = \int_\Omega f v
  \end{equation}
\end{problem}

On note $\gamma = \essinf_{x \in \Omega} (\mu -\frac{1}{2} \nabla \cdot {\bm
\beta})$, on peut alors montrer que sous la condition
\begin{equation}
  \label{eq:92}
  \alpha_0 > - \min( 0, \gamma c_\Omega  )
\end{equation}
où $c_\Omega$ est la constante de l'inégalité de Poincaré, la forme bilinéaire
$a$ est coercive et donc que, grâce au théorème de Lax-Milgram, le problème
\ref{prob:12} est bien posé. La coercivité est garantie si $\alpha_0$ est
suffisamment grand c'est à dire que si la diffusion est dominante.

L'approximation élément fini est similaire à celle du Laplacian, de plus les
variantes sur les conditions aux limites s'appliquent également: condition de
Dirichlet non homogène, de Neumann ou de Robin.



\section{Élasticité Linéaire}
\label{sec:elasticite-lineaire}

On s'intéresse dans cette section à l'approximation par éléments finis de
problèmes de mécanique des milieux continus en 3D.

Si ${\bm f}: \Omega \rightarrow \mathbb{R}^3$ est la charge extérieur
s'appliquant au domaine $\Omega$ et qu'on note $\disp: \Omega \rightarrow
\mathbb{R}^3$ le déplacement de la structure induit par cette charge ${\bm f}$
alors en supposant que les déformations soient petites pour être modélisées
dans le cadre de l'elasticité linéaire on a la relation suivante à
l'équilibre:
\begin{equation}
  \label{eq:31}
  \nabla \cdot \stresst + {\bm f} = {\bm 0} \mbox{ dans } \Omega
\end{equation}
où $\stresst: \Omega \rightarrow \R[d,d]$ est le \emph{tenseur des
contraintes} défini par la relation
\begin{equation}
  \label{eq:93}
  \stresst = \lambda \tr(\deformt) \Id[3] + 2 \mu \deformt
\end{equation}
avec $\deformt : \Omega \rightarrow \R[d,d]$ le \emph{tenseur des déformations} défini par
\begin{equation}
  \label{eq:94}
  \deformt = \frac{1}{2} \left( \nabla \disp + \nabla \disp^T \right),
\end{equation}
$\lambda$ et $\mu$ les coefficients de Lamé et $\Id[3]$ la matrice identité de
$\R[3,3]$. On a alors
\begin{equation}
  \label{eq:95}
  \stresst = \lambda( \nabla \cdot \disp ) \Id[3] + \mu( \nabla \disp + \nabla \disp^T)
\end{equation}

\paragraph{Coefficients de Lamé}
\label{sec:coefficients-de-lame}
Ils sont des coefficients phénoménologiques contraints par les relations
suivants:
\begin{itemize}
\item $\mu >0$
\item $\lambda + \frac{2}{3} \mu \ge 0$
\end{itemize}
Dans ce qui suit, on supposera que $\lambda \ge 0$ et ces coefficients
constants. D'un point de vue pratique, ces coefficients sont obtenus par les
\emph{module d'Young} $E$ et  \emph{coefficient de Poisson} $\nu$ tels que
\begin{equation}
  \label{eq:96}
  \lambda = \ds \frac{E \nu}{( 1+\nu )*( 1-2 \nu )} , \quad \mu =\frac{E}{2 ( 1+\nu )}
\end{equation}


\lstinputlisting[linerange=12-33,caption={Exemple en élasticité linéaire}]{../Codes/prudhomme/elasticite_lineaire/trous.cpp}


%%% Local Variables:
%%% coding: utf-8
%%% mode: latex
%%% TeX-PDF-mode: t
%%% TeX-parse-self: t
%%% TeX-auto-save: t
%%% x-symbol-8bits: nil
%%% TeX-auto-regexp-list: TeX-auto-full-regexp-list
%%% TeX-master: "mef-intro"
%%% ispell-local-dictionary: "american"
%%% End:
