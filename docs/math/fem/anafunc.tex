%% Preamble %%
%% A minimal LaTeX preamble
%% Some packates are needed to implement
%% Asciidoc features

\documentclass[11pt]{amsart}
\usepackage{geometry}                % See geometry.pdf to learn the layout options. There are lots.
\geometry{letterpaper}               % ... or a4paper or a5paper or ...
%\geometry{landscape}                % Activate for for rotated page geometry
%\usepackage[parfill]{parskip}       % Activate to begin paragraphs with an empty line rather than an indent

\usepackage{tcolorbox}
\usepackage{lipsum}

\usepackage{epstopdf}
\usepackage{color}
% \usepackage[usenames, dvipsnames]{color}
% \usepackage{alltt}


\usepackage{amssymb}
% \usepackage{amsmath}
\usepackage{amsthm}
\usepackage[version=3]{mhchem}


% Needed to properly typeset
% standard unicode characters:
%
\RequirePackage{fix-cm}
\usepackage{fontspec}
\usepackage[Latin,Greek]{ucharclasses}
%
% NOTE: you must also use xelatex
% as the typesetting engine


% \usepackage{fontspec}
% \usepackage{polyglossia}
% \setmainlanguage{en}

\usepackage{hyperref}
\hypersetup{
    colorlinks=true,
    linkcolor=blue,
    filecolor=magenta,
    urlcolor=cyan,
}

\usepackage{graphicx}
\usepackage{wrapfig}
\graphicspath{ {images/} }
\DeclareGraphicsExtensions{.png, .jpg, jpeg, .pdf}

%% \DeclareGraphicsRule{.tif}{png}{.png}{`convert #1 `dirname #1`/`basename #1 .tif`.png}
%% Asciidoc TeX Macros %%


% \pagecolor{black}
%%%%%%%%%%%%


% Needed for Asciidoc

\newcommand{\admonition}[2]{\textbf{#1}: {#2}}
\newcommand{\rolered}[1]{ \textcolor{red}{#1} }
\newcommand{\roleblue}[1]{ \textcolor{blue}{#1} }

\newtheorem{theorem}{Theorem}
\newtheorem{proposition}{Proposition}
\newtheorem{corollary}{Corollary}
\newtheorem{lemma}{Lemma}
\newtheorem{definition}{Definition}
\newtheorem{conjecture}{Conjecture}
\newtheorem{problem}{Problem}
\newtheorem{exercise}{Exercise}
\newtheorem{example}{Example}
\newtheorem{note}{Note}
\newtheorem{joke}{Joke}
\newtheorem{objection}{Objection}





%%%%%%%%%%%%%%%%%%%%%%%%%%%%%%%%%%%%%%%%%%%%%%%%%%%%%%%

%  Extended quote environment with author

\renewenvironment{quotation}
{   \leftskip 4em \begin{em} }
{\end{em}\par }

\def\signed#1{{\leavevmode\unskip\nobreak\hfil\penalty50\hskip2em
  \hbox{}\nobreak\hfil\raise-3pt\hbox{(#1)}%
  \parfillskip=0pt \finalhyphendemerits=0 \endgraf}}


\newsavebox\mybox

\newenvironment{aquote}[1]
  {\savebox\mybox{#1}\begin{quotation}}
  {\signed{\usebox\mybox}\end{quotation}}

\newenvironment{tquote}[1]
  {  {\bf #1} \begin{quotation} \\ }
  { \end{quotation} }

%% BOXES: http://tex.stackexchange.com/questions/83930/what-are-the-different-kinds-of-boxes-in-latex
%% ENVIRONMENTS: https://www.sharelatex.com/learn/Environments

\newenvironment{asciidocbox}
  {\leftskip6em\rightskip6em\par}
  {\par}

\newenvironment{titledasciidocbox}[1]
  {\leftskip6em\rightskip6em\par{\bf #1}\vskip-0.6em\par}
  {\par}



%%%%%%%%%%%%%%%%%%%%%%%%%%%%%%%%%%%%%%%%%%%%%%%%%%%%%%%%

%% http://texblog.org/tag/rightskip/


\newenvironment{preamble}
  {}
  {}

%% http://tex.stackexchange.com/questions/99809/box-or-sidebar-for-additional-text
%%
\newenvironment{sidebar}[1][r]
  {\wrapfigure{#1}{0.5\textwidth}\tcolorbox}
  {\endtcolorbox\endwrapfigure}


%%%%%%%%%%

\newenvironment{comment*}
  {\leftskip6em\rightskip6em\par}
  {\par}

  \newenvironment{remark*}
  {\leftskip6em\rightskip6em\par}
  {\par}


%% Dummy environment for testing:

\newenvironment{foo}
  {\bf Foo.\ }
  {}


\newenvironment{foo*}
  {\bf Foo.\ }
  {}


\newenvironment{click}
  {\bf Click.\ }
  {}

\newenvironment{click*}
  {\bf Click.\ }
  {}


\newenvironment{remark}
  {\bf Remark.\ }
  {}

\newenvironment{capsule}
  {\leftskip10em\par}
  {\par}

%%%%%%%%%%%%%%%%%%%%%%%%%%%%%%%%%%%%%%%%%%%%%%%%%%%%%

%% Style

\parindent0pt
\parskip8pt
%% User Macros %%
%% Front Matter %%

\title{Outils d’analyse fonctionnelle}
\author{}
\date{}


%% Begin Document %%

\begin{document}
\maketitle
\hypertarget{x-outils-d’analyse-fonctionnelle}{\section*{Outils d’analyse fonctionnelle}}
\hypertarget{x-quelques-rappels}{\subsection*{Quelques rappels}}
\hypertarget{x-normes-et-produits-scalaires}{\subsubsection*{Normes et produits scalaires}}
Soit $E$ un espace vectoriel. \\


[def:7] $\|.\|$ : $E \rightarrow \RR$ est une
\textbf{norme} sur $E$ ssi elle vérifie :


\begin{description}

\item[$\qquad$(N1)]$\left( \| x \| = 0 \right)  \Longrightarrow (x=0)$

\item[$\qquad$(N2)]$\forall\, \lambda\in\RR,\; \forall x\in E, \quad \| \lambda x \|  = |\lambda| \; \| x \| $

\item[$\qquad$(N3)]$\forall\,  x,y \in E, \quad \| x+ y \| \le \|x \| + \|y\|\qquad$
(inégalité triangulaire) \\

\end{description}


Pour $E=\RR^n$ et $x=(x_1,\ldots,x_n) \in\RR^n$,
on définit les normes \| x \|_1 = \sum_{i=1}^n |x_i|
\qquad
\| x \|_2 = \left( \sum_{i=1}^n x_i^2 \right)^{1/2}
\qquad
\| x \|_\infty = \sup_{i} |x_i|]


On appelle \textbf{produit scalaire} sur $E$ toute forme bilinéaire
symétrique définie positive. \
$\quad<.,.>$ : $E\times E \rightarrow \RR$ est
donc un produit scalaire sur $E$ ssi il vérifie :


\begin{description}

\item[$\qquad$(S1)]$\forall\; x,y \in E, \quad <x,y> = <y,x>$

\item[$\qquad$(S2)]$\forall\; x_1,x_2,y \in E, \quad <x_1+x_2,y> = <x_1,y>
    + <x_2,y> $

\item[$\qquad$(S3)]$\forall\; x,y \in E, \, \forall\, \lambda\in\RR,\quad
    <\lambda x,y> = \lambda <x,y> $

\item[$\qquad$(S4)]$\forall\;  x \in E, x\ne 0, \quad  <x,x>\; > 0 $ \

\end{description}


A partir d’un produit scalaire, on peut définir une \textbf{norme induite} :
$ \| x \| = \sqrt{<x,x>} $ \
On a alors, d’après (N3), l’\textbf{inégalité de Cauchy-Schwarz} :
$\ds{ | <x,y> | \le \| x \| \; \| y \| }$


\textbf{Exemple :} Pour $E=\RR^n$, on définit le produit scalaire
$\ds{<x,y> = \sum_{i=1}^n x_i \, y_i}$. Sa norme induite est
$\| . \|_2$ définie précédemment.


Un espace vectoriel muni d’une norme est appelé \textbf{espace normé}. \
Un espace vectoriel muni d’un produit scalaire est appelé \textbf{espace
préhilbertien}. En particulier, c’est donc un espace normé pour la norme
induite.


\hypertarget{x-suites-de-cauchy---espaces-complets}{\subsubsection*{Suites de Cauchy - espaces complets}}
[def:1] Soit $E$ un espace vectoriel et
$(x_n)_n$ une suite de $E$.
$(x_n)_n$ est une \textbf{suite de Cauchy} ssi
$\forall \varepsilon > 0,\;\; \exists N / \forall p>N, \forall q>N, \quad \|x_p - x_q \| < \varepsilon$


Toute suite convergente est de Cauchy. La réciproque est fausse.


[def:2] Un espace vectoriel est \textbf{complet} ssi toute suite de Cauchy y
est convergente.


[def:3] Un espace normé complet est un \textbf{espace de Banach}.


[def:4] Un espace préhilbertien complet est un \textbf{espace de Hilbert}. \


[def:5] Un espace de Hilbert de dimension finie est appelé \textbf{espace
euclidien}.


\hypertarget{x-espaces-fonctionnels}{\subsection*{Espaces fonctionnels}}
[def:6] Un \textbf{espace fonctionnel} est un espace vectoriel dont les
éléments sont des fonctions.


${\cal C}^p([a;b)$] désigne l’espace des fonctions définies
sur l’intervalle $[a,b$], dont toutes les dérivées jusqu’à
l’ordre $p$ existent et sont continues sur
$[a,b$].


Dans la suite, les fonctions seront définies sur un sous-ensemble de
$\RR^n$ (le plus souvent un ouvert noté
$\Omega$), à valeurs dans $\RR$ ou
$\RR^p$.


La température $T(x,y,z,t)$ en tout point d’un objet
$\bar{\Omega}\subset \RR^3$ est une fonction de
$ \bar{\Omega} \times \RR \longrightarrow \RR$.


Les normes usuelles les plus simples sur les espaces fonctionnels sont
les \textbf{normes} $\bf L^p$ définies par :
\[\| u \|_{L^p} = \left ( \int_{\Omega } |u|^p \right) ^{1/p} \quad ,\; p\in [1,+ \infty[ ,
\qquad
\hbox{et}\qquad
\| u \|_{L^\infty} = {\hbox{Sup}}_{\Omega } |u|] Comme on va le voir,
ces formes $L^p$ ne sont pas nécessairement des normes. Et
lorsqu’elles le sont, les espaces fonctionnels munis de ces normes ne
sont pas nécessairement des espaces de Banach. Par exemple, les formes
$L^\infty$ et $L^1$ sont bien des normes sur
l’espace ${\cal C}^0([a;b)$], et cet espace est complet si
on le munit de la norme $L^\infty$, mais ne l’est pas si on
le munit de la norme $L^1$. Pour cette raison, on va définir
les espaces ${\cal L}^p(\Omega)$
($p\in [1,+ \infty[$) par
\[{\cal L}^p(\Omega) = \left\{  u : \Omega \rightarrow \RR, \hbox{ mesurable, et telle que } \int_\Omega |u|^p<\infty  \right\}]
( on rappelle qu’une fonction $u$ est mesurable ssi
$\{ x /  |u(x)|<r \}$ est mesurable
$\forall r>0$. ) Sur ces espaces
${\cal L}^p(\Omega)$, les formes $L^p$ ne sont
pas des normes. En effet, $\| u \|_{L^p} = 0$ implique que
$u$ est nulle presque partout dans
${\cal L}^p(\Omega)$, et non pas $u=0$. C’est
pourquoi on va définir les \textbf{espaces} $\bf L^p(\Omega)$ :


$L^p(\Omega)$ est la classe d’équivalence des fonctions de
${\cal
  L}^p(\Omega)$ pour la relation d’équivalence ``égalité presque
partout''. Autrement dit, on confondra deux fonctions dès lors qu’elles
sont égales presque partout, c’est à dire qu’elles ne diffèrent que sur
un ensemble de mesure nulle.[def:9]


La forme $L^p$ est une norme sur $L^p(\Omega)$,
et $L^p(\Omega)$ muni de la norme $L^p$ est un
espace de Banach (c.a.d. est complet).[thr:1]


Un cas particulier très important est $p=2$. On obtient
alors l’\textbf{espace fonctionnel $L^2(\Omega)$}, c’est à dire
l’espace des fonctions de carré sommable sur $\Omega$ (à la
relation d’équivalence ``égalité presque partout'' près). A la norme
$L^2$ :
$\| u \|_{L^2} = \left( \int_\Omega u^2 \right)^{1/2} $, on
peut associer la forme bilinéaire
$(u,v)_{L^2} = \int_\Omega u\, v$. Il s’agit d’un produit
scalaire, dont dérive la norme $L^2$. D’où :


$L^2(\Omega)$ est un espace de Hilbert.[thr:2]


\hypertarget{x-notion-de-dérivée-généralisée}{\subsection*{Notion de dérivée généralisée}}
Nous venons de définir des espaces fonctionnels complets, ce qui sera un
bon cadre pour démontrer l’existence et l’unicité de solutions
d’équations aux dérivées partielles, comme on le verra plus loin
notamment avec le théorème de Lax-Milgram. Toutefois, on a vu que les
éléments de ces espaces $L^p$ ne sont pas nécessairement des
fonctions très régulières. Dès lors, les dérivées partielles de telles
fonctions ne sont pas forcément définies partout. Pour s’affranchir de
ce problème, on va étendre la notion de dérivation. Le véritable outil à
introduire pour cela est la notion de \textbf{distribution}, due à L. Schwartz
(1950). Par manque de temps dans ce cours, on se contentera ici d’en
donner une idée très simplifiée, avec la notion de \textbf{dérivée
généralisée}. Cette dernière a des propriétés beaucoup plus limitées que
les distributions, mais permet de “sentir" les aspects nécessaires pour
mener à la formulation variationnelle. Dans la suite,
$\Omega$ sera un ouvert (pas nécessairement borné) de
$\RR^n$.


\hypertarget{x-fonctions-tests}{\subsubsection*{Fonctions tests}}
Soit $\varphi : \Omega \rightarrow \RR$. On appelle \textbf{support
de $\bf
  \varphi$} l’adhérence de
$\{ x \in \Omega / \varphi(x) \ne 0 \}$.[def:10]


Pour $\Omega = -1,1[$], et $\varphi$ la fonction
constante égale à 1, $\hbox{Supp}\, \varphi = [-1,1$].


On note ${\cal D}(\Omega)$ l’espace des fonctions de
$\Omega$ vers $\RR$, de classe
${\cal C}^\infty$, et à support compact inclus dans
$\Omega$. ${\cal D}(\Omega)$ est parfois appelé
\textbf{espace des fonctions-tests}.[def:11]


L’exemple le plus classique dans le cas 1-D est la fonction (x) = \{


ll & |x|<1 \
0 & |x|1 \


\begin{enumerate}

\item{[eq:fonction-test1] $\varphi$ est une fonction de
${\cal D}(a,b[)$] pour tous $a < -1 < 1 < b$.}

\end{enumerate}


Cet exemple s’étend aisément au cas multi-dimensionnel
($n>1$). Soit $a\in\Omega$ et $r>0$
tel que la boule fermée de centre $a$ et de rayon
$r$ soit incluse dans $\Omega$. On pose alors :
(x) = \{


ll & |x-a|<r \
0 & \


\begin{enumerate}

\item{[eq:fonction-test2] $\varphi$ ainsi définie est élément de
${\cal D}(\Omega)$.}

\end{enumerate}


$\overline{{\cal D}(\Omega) } = L^2(\Omega)$[thr:4]


\hypertarget{x-dérivée-généralisée}{\subsubsection*{Dérivée généralisée}}
Soit $u\in {\cal C}^1(\Omega)$ et
$\varphi \in {\cal D}(\Omega)$. Par intégration par parties
(annexe [sec:green]), on a :
\[\int_\Omega \partial_i u\;  \varphi = - \int_\Omega u \; \partial_i\varphi + \int_{\partial \Omega} u \; \varphi \; {\bf e}_i.{\bf n}]
Ce dernier terme (intégrale sur le bord de $\Omega$) est nul
car $\varphi$ est à support compact (donc nul sur
$\partial \Omega$). Or $\int_\Omega u
\; \partial_i\varphi$ a un sens par exemple dès que
$u\in L^2(\Omega)$. Donc le terme
$\int_\Omega \partial_i u\; \varphi$ a aussi du sens, sans
que $u$ ne soit nécessairement de classe
${\cal C}^1$. Ceci permet de définir
$\partial_i u$ même dans ce cas.


cas 1-D $\quad$ Soit $I$ un intervalle de , pas
forcément borné. On dit que $u\in L^2(I)$ admet une \textbf{dérivée
généralisée} dans $L^2(I)$ ssi
$\exists u_1\in L^2(I)$ telle que
$\forall \varphi\in {\cal
  D}(I), \quad \int_I u_1\;\varphi = - \int_I u \varphi'$[def:12]


Soit $I=a,b[$] un intervalle borné, et $c$ un
point de $I$. On considère une fonction $u$
formée de deux branches de classe ${\cal C}^1$, l’une sur
$a,c[$], l’autre sur $c,b[$], et se raccordant
de façon continue mais non dérivable en $c$. Alors
$u$ admet une dérivée généralisée définie par
$u_1(x)=u'(x)\quad \forall x\ne c$. En effet :
\[\forall \varphi\in {\cal D}(a,b[)\qquad \int_a^b u \varphi' = \int_a^c \
\int_c^b = - \int_a^c u' \varphi - \int_c^b u'\varphi \
\underbrace{(u(c${}^{-)-u(c}$+))}_{=0} \, \varphi(c)] par intégration par
parties. La valeur $u_1(c)$ n’a pas d’importance: on a de
toute façon au final la même fonction de $L^2(I)$,
puisqu’elle est définie comme classe d’équivalence de la relation
d’équivalence “égalité presque partout".


En itérant, on dit que $u$ admet une \textbf{dérivée généralisée
d’ordre $\bf k$} dans $L^2(I)$, notée
$u_k$, ssi $\ds{\forall \varphi\in
  {\cal D}(I), \quad \int_I u_k\;\varphi = (- 1)^k \; \int_I u \varphi^{(k)}
  }$[def:13]


Ces définitions s’étendent naturellement pour la définition de dérivées
partielles généralisées, dans le cas $n>1$.


Quand elle existe, la dérivée généralisée est unique.[thr:5]


Quand $u$ est de classe
${\cal C}^1(\bar{\Omega})$, la dérivée généralisée est égale
à la dérivée classique.[thr:6]


\hypertarget{x-espaces-de-sobolev}{\subsection*{Espaces de Sobolev}}
Les espaces $H^m$
${}^{^}$${}^{^}$${}^{^}$${}^{^}$${}^{^}$${}^{^}$${}^{^}$${}^{^}$${}^{^}$^^


$\ds{ H^1(\Omega) = \left\{ u \in L^2(\Omega)\; / \; \partial_i u \; \in
    L^2(\Omega), \quad 1 \le i \le n \right\} }$ où
$\partial_i u$ est définie au sens de la dérivée
généralisée.[def:14]


$H^1(\Omega)$ est appelé \textbf{espace de Sobolev d’ordre 1}.


Pour tout entier $m\ge 1$,
H^m(\Omega) = \left\{ u \in L^2(\Omega) \; / \; \partial^\alpha u \; \in
  L^2(\Omega) \quad \forall \alpha =(\alpha_1,\ldots,\alpha_n) \in \NN^n\hbox{
  tel que}\; |\alpha|= \alpha_1+\cdots+\alpha_n \le m \right\}][def:15]


$H^m(\Omega)$ est appelé \textbf{espace de Sobolev d’ordre
$\bf m$}. Par extension, on voit aussi que
$H^0(\Omega)=L^2(\Omega)$. Dans le cas de la dimension 1, on
écrit plus simplement pour $I$ ouvert de $\RR$ :
\[H^m(I) =  \left\{ u \in L^2(I)  \; / \;   u', \ldots, u^{(m)} \in L^2(I) \right\}]


$H^1(\Omega)$ est un espace de Hilbert pour le produit
scalaire
\[(u,v)_1 = \int_\Omega u \, v\, + \sum_{i=1}^n \; \int_\Omega \partial_i u
\; \partial_i v = (u,v)_0 + \sum_{i=1}^n (\partial_i u, \partial_i v )_0]
en notant $(.,.)_0$ le produit scalaire $L^2$.
On notera $\|.\|_1$ la norme associée à
$(.,.)_1$.[thr:7]


On définit de même un produit scalaire et une norme sur
$H^m(\Omega)$ par
\[(u,v)_m =   \sum_{|\alpha| \le m} ( \partial^\alpha u , \partial^\alpha v )_0 \qquad
\hbox{ et }\qquad
\| u \|_m = (u,u)_m^{1/2}]


$H^m(\Omega)$ muni du produit scalaire $(.,.)_m$
est un espace de Hilbert.[thr:8]


Si $\Omega$ est un ouvert de $\RR^n$ de
frontière $\partial\Omega$ “suffisamment régulière" (par
exemple ${\cal C}^1$), on a l’inclusion :
$H^m(\Omega) \subset {\cal C}^k(\bar{\Omega})$ pour
$\ds{ k < m-\frac{n}{2}
  }$[thr:9]


En particulier, on voit que pour un intervalle $I$ de
$\RR$, on a $H^1(I)
  \subset {\cal C}^0(\bar{I})$, c’est à dire que, en 1-D, toute
fonction $H^1$ est continue.


L’exemple de $u(x) = x\, \sin\frac{1}{x}$ pour
$x\in0,1$]] et $u(0)=0$ montre que la réciproque
est fausse.


L’exemple de $u(x,y) = | \ln (x^2+y^2) |^k$ pour
$0<k<1/2$ montre qu’en dimension supérieure à 1 il existe
des fonctions $H^1$ discontinues.


\hypertarget{x-trace-d’une-fonction}{\subsubsection*{Trace d’une fonction}}
Pour pouvoir faire les intégrations par parties qui seront utiles par
exemple pour la formulation variationnelle, il faut pouvoir définir le
prolongement (\emph{la trace}) d’une fonction sur le bord de l’ouvert
$\Omega$. : on considère un intervalle ouvert
$I=a,b[$] borné. On a vu que
$H^1(I) \subset {\cal C}^0(\bar{I})$. Donc, pour
$u\in H^1(I)$, $u$ est continue sur
$[a,b$], et $u(a)$ et $u(b)$ sont
bien définies. : on n’a plus
$H^1(\Omega) \subset {\cal C}^0(\bar{\Omega})$. Comment
alors définir la trace ? La démarche est la suivante :


\begin{itemize}

\item On définit l’espace
\[{\cal C}^1(\bar{\Omega}) = \left\{  \varphi : \Omega \rightarrow \RR \;/\;  \exists O \hbox{ ouvert contenant } \bar{\Omega},\; \exists \psi \in {\cal C}^1(O),\; \psi_{|\Omega} = \varphi \right\}]
Autrement dit, ${\cal C}^1(\bar{\Omega})$ est l’espace des
fonctions ${\cal C}^1$ sur $\Omega$,
prolongeables par continuité sur $\partial\Omega$ et dont le
gradient est lui-aussi prolongeable par continuité. Il n’y a donc pas de
problème pour définir la trace de telles fonctions.

\item On montre que, si $\Omega$ est un ouvert borné de
frontière $\partial\Omega$ “assez régulière", alors
${\cal C}^1(\bar{\Omega})$ est dense dans
$H^1(\Omega)$.

\item L’application linéaire continue, qui à toute fonction $u$
de ${\cal C}^1(\bar{\Omega})$ associe sa trace sur
$\partial\Omega$, se prolonge alors en une application
linéaire continue de $H^1(\Omega)$ dans
$L^2(\partial\Omega)$, notée $\gamma_0$, qu’on
appelle \textbf{application trace}. On dit que $\gamma_0(u)$ \textbf{est
la trace de $u$ sur} $\partial\Omega$.

\end{itemize}


Pour une fonction $u$ de $H^1(\Omega)$ qui soit
en même temps continue sur $\bar{\Omega}$, on a évidemment
$\gamma_0(u) = u_{|\partial\Omega}$. C’est pourquoi on note
souvent par abus simplement $u_{|\partial\Omega}$ plutôt que
$\gamma_0(u)$.


On peut de façon analogue définir $\gamma_1$, application
trace qui permet de prolonger la définition usuelle de la dérivée
normale sur $\partial\Omega$. Pour
$u\in H^2(\Omega)$, on a
$\partial_i u \in H^1(\Omega)$,
$\forall i=1,\ldots,n$, et on peut donc définir
$\gamma_0(\partial_i u)$. La frontière
$\partial\Omega$ étant “assez régulière" (par exemple,
idéalement, de classe ${\cal C}^1$), on peut définir la
normale
$n=\left(   \begin{array}{l}  n_1 \ \vdots \ n_n \end{array} \right)$
en tout point de $\partial\Omega$. On pose alors
$\ds{\gamma_1(u) = \sum_{i=1}^n \gamma_0(\partial_i u) n_i}$.
Cette application continue $\gamma_1$ de
$H^2(\Omega)$ dans $L^2(\partial\Omega)$ permet
donc bien de prolonger la définition usuelle de la dérivée normale. Dans
le cas où $u$ est une fonction de $H^2(\Omega)$
qui soit en même temps dans ${\cal C}^1(\bar{\Omega})$, la
dérivée normale au sens usuel de $u$ existe, et
$\gamma_1(u)$ lui est évidemment égal. C’est pourquoi on
note souvent, par abus, $\partial_n u$ plutôt que
$\gamma_1(u)$.


Espace $\bf H^1_0(\Omega)$
${}^{^}$${}^{^}$${}^{^}$${}^{^}$${}^{^}$${}^{^}$${}^{^}$${}^{^}$${}^{^}$${}^{^}$${}^{^}$${}^{^}$^^


Soit $\Omega$ ouvert de $\RR^n$. L’espace
$H^1_0(\Omega)$ est défini comme l’adhérence de
${\cal D}(\Omega)$ pour la norme $\|.\|_1$ de
$H^1(\Omega)$. (on rappelle que
${\cal D}(\Omega)$ est l’espace des fonctions
${\cal C}^\infty$ sur $\Omega$ à support
compact, encore appelé espace des fonctions tests)[def:16]


Par construction $H^1_0(\Omega)$ est un espace complet.
C’est un espace de Hilbert pour la norme $\|.\|_1$[thr:10]


: on considère un intervalle ouvert $I=a,b[$] borné. Alors
\[H^1_0(a,b[) = \left\{ u \in H^1(]a,b[),\; u(a)=u(b)=0 \right\}]
: Si $\Omega$ est un ouvert borné de frontière“assez
régulière" (par exemple ${\cal C}^1$ par morceaux), alors
$H^1_0(\Omega) = \ker \gamma_0$. $H^1_0(\Omega)$
est donc le sous-espace des fonctions de $H^1(\Omega)$ de
trace nulle sur la frontière $\partial\Omega$.


Pour toute fonction $u$ de $H^1(\Omega)$, on
peut définir :
\[\ds{ |u|_1 = \left( \sum_{i=1}^n \| \partial_i u \|_0^2 \right)^{1/2}
= \left( \int_\Omega \sum_{i=1}^n \left( \partial_i u \right)^2 dx
\right)^{1/2} } \vspace*{5 mm}][def:17]


[thr:11] Si $\Omega$ est borné dans au moins une direction,
alors il existe une constante $C(\Omega)$ telle que
$\forall u \in H^1_0(\Omega), \; \|u\|_0 \le
  C(\Omega)\; |u|_1$.


On en déduit que $|.|_1$ est une norme sur
$H^1_0(\Omega)$, équivalente à la norme
$\|.\|_1$.


Le résultat précédent s’étend au cas où l’on a une condition de
Dirichlet nulle seulement sur une partie de
$\partial\Omega$, si $\Omega$ est connexe.


On suppose que $\Omega$ est un ouvert borné connexe, de
frontière ${\cal
C}^1$ par morceaux. Soit
$V=\left\{ v\in H^1(\Omega),\, v=0 \hbox{ sur
  }\Gamma_0 \right\}$ où latexmath:[$\Gamma_0$] est une partie de
$\partial\Omega$ de mesure non-nulle. Alors il existe une
constante $C(\Omega)$ telle que $\forall
u \in V, \; \|u\|_{0,V} \le C(\Omega)\; |u|_{1,V}$, où
$\|.\|_{0,V}$ et $|.|_{1,V}$ désignent les norme
et semi-norme induites sur $V$. On en déduit que
$|.|_{1,V}$ est une norme sur $V$, équivalente à
la norme $\|.\|_{1,V}$.   \\


\hypertarget{x-exercices}{\subsubsection*{Exercices}}
\begin{enumerate}

\item{Montrer que les fonctions définies par ([eq:fonction-test1]) et
([eq:fonction-test2]) sont bien ${\cal C}^\infty$ à support
compact.}

\item{Montrer que ${\cal C}^0([a,b)$] est un espace complet
pour la norme $L^\infty$.}

\item{Montrer que ce n’est pas le cas pour la norme $L^1$
(exhiber une suite de Cauchy non convergente dans
${\cal C}^0([a,b)$]).}

\item{Démontrer que, lorsqu’elle existe, la dérivée généralisée est
unique.}

\item{Démontrer que, pour une fonction de classe ${\cal C}^1$,
la dérivée généralisée est égale à la dérivée classique.}

\item{Soit une fonction de $[a,b$] vers $\RR$,
formée de deux branches de classe ${\cal C}^1$ sur
$[a,c[$ et $c,b$]], et discontinue en
$c$. Montrer qu’elle n’admet pas de dérivée généralisée. (il
faudrait alors avoir recours à la notion de distribution pour dériver
cette fonction).}

\item{Montrer que $|.|_1$ est une norme sur
$H^1_0(\Omega)$, équivalente à la norme
$\|.\|_1$}

\end{enumerate}


\end{document}

