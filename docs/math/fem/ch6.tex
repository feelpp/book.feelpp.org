\chapter{Quelques aspects pratiques de la m\'ethode des \'el\'ements finis}
%
\noindent
%
On va donner dans ce chapitre quelques indications pratiques concernant la
programmation d'une m\'ethode d'\'el\'ements finis. On  pourra trouver beaucoup
plus de d\'etails par exemple dans les ouvrages de Joly (1990) et de Lucquin et
Pironneau (1996).\saut
%
On exposera ici quelques principes g\'en\'eraux dans le cas classique d'une m\'ethode d'\'el\'ements finis $P_1$ de Lagrange sur un maillage triangulaire d'un domaine de $\RR^2$. Des modifications devront donc \^etre apport\'ees pour d'autres types d'\'el\'ements finis.
%
%
\section{Maillage}
\noindent
%
%
On va se placer ici dans le cas fr\'equent de la r\'esolution d'un probl\`eme sur un ouvert born\'e de $\RR^2$, not\'e $\Omega$. Ce probl\`eme comporte des conditions aux limites sur $\partial \Omega$, qui peuvent \^etre de type Dirichlet ou Neumann. On notera $\Gamma_0$ la partie de $\partial \Omega$ o\`u ces conditions sont de type Dirichlet, et $\Gamma_1$ la partie o\`u elles sont de type Neumann. On aura $\Gamma_0 \cap \Gamma_1=\emptyset$ et $\partial \Omega=\Gamma_0 \cup \Gamma_1$.\saut
%
%
Comme dans les chapitres pr\'ec\'edents, on va noter $N_e$ le nombre d'\'el\'ements du maillage, et $N_h$ le nombre de noeuds (qui sont ici simplement les sommets des triangles).\saut
%
Le rep\'erage des noeuds est fait par l'interm\'ediaire d'un tableau COOR$(2,N_h)$ :
\begin{center}
\begin{tabular}{ll}
 COOR$(1,k)=$ & abscisse du noeud $k$\\
 COOR$(2,k)=$ & ordonn\'ee du noeud $k$
\end{tabular}
\end{center}
%
C'est le seul tableau travaillant avec les coordonn\'ees ``physiques'' dans le domaine. Tous les autres seront des tableaux d'adressage relatif.\saut
%
La d\'efinition des \'el\'ements du maillage est faite par un tableau CONEC$(3,N_e)$, qui fait le lien noeuds-\'el\'ements. Chaque \'el\'ement contient 3 noeuds ``locaux'' (car on travaille dans cet exemple sur des triangles). Ces noeuds correspondent chacun \`a un indice du tableau $COOR$, qu'on appellera leur ``num\'ero global''.\\
%
\hspace*{2 cm} CONEC$(i,l)=$ num\'ero global du $i^{\hbox{\tiny \`eme}}$ noeud local de l'\'el\'ement $l$ ($i=1,2,3$)\vspace*{5 mm}\\
%
Le rep\'erage des conditions de Dirichlet est r\'ealis\'e directement par un tableau DIRI$(N_0)$ balayant les $N_0$ noeuds de $\Gamma_0$ :\\
%
\hspace*{2 cm} DIRI$(i)=$ num\'ero global du $i^{\hbox{\tiny \`eme}}$ noeud de $\Gamma_0$.\saut
%
Le rep\'erage des conditions de Neumann est r\'ealis\'e  par un premier tableau NEUM$_0(N_1)$ balayant les $N_1$ \'el\'ements ayant un c\^ot\'e sur $\Gamma_1$, et par un second tableau NEUM$(3,N_1)$ indiquant quels c\^ot\'es des \'el\'ements sont sur $\Gamma_1$ :
%
\begin{center}
\begin{tabular}{lp{10 cm}}
 NEUM$_0(j)=$ & num\'ero du $j^{\hbox{\tiny \`eme}}$ \'el\'ement ayant un c\^ot\'e sur $\Gamma_1$.\\
%
NEUM$(i,j)=$ & 1 si le c\^ot\'e oppos\'e au $i^{\hbox{\tiny \`eme}}$ noeud local du $j^{\hbox{\tiny \`eme}}$ \'el\'ement de la liste NEUM$_0$ est sur $\Gamma_1$, 0 sinon.
%
\end{tabular}
\end{center}
%
%
\section{Assemblage de la matrice du syst\`eme}
\noindent
%
%
Le syst\`eme lin\'eaire auquel aboutit la d\'emarche des \'el\'ements finis est 
$A\mu=b$, avec
\begin{eqnarray*}
A_{ij} & = & a(\varphi_i,\varphi_j) = \int_\Omega \cdots \; + \int_{\Gamma_1} \cdots\\
b_{j} & = & l(\varphi_j) = \int_\Omega \cdots \; + \int_{\Gamma_1} \cdots\\
\end{eqnarray*}
%
Les \'etapes de la programmation sont alors :
\begin{enumerate}
\item Calcul de $A_{ij}=\int_\Omega \cdots$ , pour $i=1,\ldots,N_h$, $j=1,\ldots,N_h$.
%
\item Calcul de $b_j=\int_\Omega \cdots$ , pour $j=1,\ldots,N_h$.
%
\item Conditions de Neumann : \\
Pour tous les noeuds $j$ des \'el\'ements ayant un c\^ot\'e sur $\Gamma_1$ faire:
\begin{itemize}
\item $\ds{A_{ij}=A_{ij} + \int_{\Gamma_1} \cdots}\;$ pour les noeuds $i$ des \'el\'ements ayant un c\^ot\'e sur $\Gamma_1$
\item $\ds{b_j=b_j + \int_{\Gamma_1} \cdots }$
\end{itemize}
%
\item Conditions de Dirichlet : on modifie les composantes de $A$ et $b$ o\`u les noeuds de $\Gamma_0$ interviennent.\\
Pour tous les noeuds $j$ de $\Gamma_0$ faire:
\begin{itemize}
\item $b_i=b_i-u_j A_{ij}$ pour tous les noeuds $i$ de $\Omega \setminus \Gamma_0$ ($u_j$ d\'esigne la valeur impos\'ee sur le noeud $j$ par les conditions de Dirichlet)
\item $A_{ij}=A_{ji}=0$ pour tous les noeuds $i$ de $\Omega \setminus \Gamma_0$
\item $A_{ij}=0$ pour tous les noeuds $i$ de $\Gamma_0$
\item $A_{jj}=1$
\item $b_j=u_j$
\end{itemize}
%
\vspace*{10 mm}
\end{enumerate}
%
%
L'\'etape la plus co\^uteuse est la premi\`ere, c'est \`a dire le calcul de $\ds{A_{ij}=\int_\Omega H(\varphi_i,\varphi_j)}$, o\`u $H$ est un op\'erateur d\'ependant du probl\`eme que l'on traite. $A_{ij}$ peut \'evidemment \^etre d\'ecompos\'e sur les \'el\'ements du maillage :
$$
A_{ij} = \sum_{l=1}^{N_l} A_{ij}^{(l)} \qquad\hbox{ avec } A_{ij}^{(l)}= \int_{K_l} H(\varphi_i,\varphi_j)
$$
%
Une m\'ethode na\"{\i}ve de calcul serait :
\begin{center}
\begin{minipage}{10 cm}
{\tt 
Pour $i=1$ \`a $N_h$\\
Pour $j=1$ \`a $N_h$\\
\hspace*{5 mm}Pour $l=1$ \`a $N_l$\\
\hspace*{10 mm}Calcul de $A_{ij}^{(l)}$\\
\hspace*{10 mm}$A_{ij}=A_{ij}+A_{ij}^{(l)}$\\
\hspace*{5 mm}Fin pour\\
Fin pour\\
Fin pour
}\\
\end{minipage}
\end{center}
%
%
Toutefois, on calcule ainsi une grande majorit\'e de contributions $A_{ij}^{(l)}$ nulles. En effet, $A_{ij}^{(l)}\ne 0$ ssi les noeuds $i$ et $j$ appartiennent \`a l'\'el\'ement $l$. On peut donc reprendre l'assemblage de $A$
en bouclant cette fois sur les \'el\'ements, pour ne calculer que les termes utiles:
%
\begin{center}
\begin{minipage}{10 cm}
{\tt 
Pour $l=1$ \`a $N_l$\\
\hspace*{5 mm}Pour $i_0=1$ \`a $3$\\
\hspace*{5 mm}Pour $j_0=1$ \`a $3$\\
\hspace*{10 mm}$i=$ CONEC$(i_0,l)$\\
\hspace*{10 mm}$j=$ CONEC$(j_0,l)$\\
\hspace*{10 mm}Calcul de $A_{ij}^{(l)}$\\
\hspace*{10 mm}$A_{ij}=A_{ij}+A_{ij}^{(l)}$\\
\hspace*{5 mm}Fin pour\\
\hspace*{5 mm}Fin pour\\
Fin pour
}\\
\end{minipage}
\end{center}
%
%
On voit que la m\'ethode d'assemblage na\"{\i}ve conduit \`a $N_h^2\times N_l$ calculs \'el\'ementaires, contre $9 N_l$ pour la seconde m\'ethode. Dans le cas r\'eel d'un maillage \`a $N_l=10^6$ \'el\'ements, on aura environ $N_h=5\, 10^5$ noeuds, ce qui m\`ene \`a $2.5\, 10^{17}$ calculs \'el\'ementaires pour la m\'ethode na\"{\i}ve, contre $9\, 10^6$ pour la seconde m\'ethode !!
%
%
\section{Formules de quadrature}
%
%
\subsection{D\'efinitions}
\noindent
%
%
Le calcul des int\'egrales sur les \'el\'ements du maillage a souvent lieu par int\'egration num\'erique. On utilise pour cela une formule de quadrature, c'est \`a dire une formule du type 
\be
\int_K f(x) \; dx \simeq \sum_{m=1}^M \omega_m f(\xi_m)
\label{eq:quadra}
\ee
o\`u les  $\xi_m$  sont des points de $K$ (appel\'es points de quadrature) et les $\omega_m$ des coefficients de pond\'eration (ou encore des poids). On choisit en g\'en\'eral ces formules de telle sorte qu'elles soient exactes pour les polyn\^omes jusqu'\`a un certain degr\'e. On a d'ailleurs le r\'esultat suivant :\vspace*{3 mm}\\
%
{\bf Th\'eor\`eme :} Si la formule (\ref{eq:quadra}) est exacte pour les polyn\^omes de degr\'e inf\'erieur ou \'egal \`a $r$, alors il existe une constante $C>0$ telle que, pour toute fonction $f\in {\cal C}^{r+1}(K)$ :
\be
\left| \int_K f(x)\; dx - \sum_{m=1}^M \omega_m f(\xi_m) \right| \le C \; |K| \; h^{r+1}
\label{eq:quadrature}
\ee
o\`u $|K|$ est la mesure de $K$ et $h$ son diam\`etre.
%
%
\subsection{Quadrature en 1-D}
\noindent
%
%
On cherche \`a estimer $\ds{\int_a^b f(x)\; dx}$. Les formules les plus courantes sont :\vspace*{3 mm}\\
\begin{description}
\item[formule des rectangles]
\be
\int_a^b f(x)\; dx \simeq (b-a) \, f(a)
\ee
Elle est exacte pour les polyn\^omes constants.\\
%
\item[formule du point milieu]
\be
\int_a^b f(x)\; dx \simeq (b-a) \, f\left( \frac{a+b}{2} \right)
\ee
Elle est exacte pour les polyn\^omes de degr\'e inf\'erieur ou \'egal \`a 1.\\
%
\item[formule des trap\`ezes]
\be
\int_a^b f(x)\; dx \simeq (b-a) \, \frac{f(a)+f(b)}{2}
\ee
Elle est exacte pour les polyn\^omes de degr\'e inf\'erieur ou \'egal \`a 1.\\
%
\item[formule \`a deux points internes]
\be
\int_a^b f(x)\; dx \simeq (b-a) \,  \frac{f(\xi_1)+f(\xi_2)}{2}
\ee
avec $\xi_1=a+\lambda (b-a)$, $\xi_2=b-\lambda (b-a)$ et $\ds{\lambda=\frac{\sqrt{3}-1}{2\, \sqrt{3}} }$.
Elle est exacte pour les polyn\^omes de degr\'e inf\'erieur ou \'egal \`a 2.\\
%
\item[formule de Simpson]
\be
\int_a^b f(x)\; dx \simeq (b-a) \, \frac{f(a)+4f(\frac{a+b}{2})+f(b)}{6}
\ee
Elle est exacte pour les polyn\^omes de degr\'e inf\'erieur ou \'egal \`a 3.
%
\end{description}
%
%
%
\subsection{Quadrature en 2-D triangulaire}
\noindent
%
%
$K$ est cette fois un triangle, de sommets $a_1,a_2,a_3$, de milieux des ar\^etes $a_{12}, a_{13}, a_{23}$, de centre de gravit\'e $a_0$, et de surface $|K|$.
%
\begin{description}
\item[formule centr\'ee]
\be
\int_K f(x)\; dx \simeq |K| \, f(a_0)
\ee
Elle est exacte pour les polyn\^omes de degr\'e inf\'erieur ou \'egal \`a 1.\\
%
\item[formule sur les sommets]
\be
\int_K f(x)\; dx \simeq \frac{|K|}{3} \, \left( f(a_1)+f(a_2)+f(a_3) \right)
\ee
Elle est exacte pour les polyn\^omes de degr\'e inf\'erieur ou \'egal \`a 1.\\
%
\item[formule sur les milieux]
\be
\int_K f(x)\; dx \simeq \frac{|K|}{3} \, \left( f(a_{12})+f(a_{23})+f(a_{13}) \right)
\ee
Elle est exacte pour les polyn\^omes de degr\'e inf\'erieur ou \'egal \`a 2.
\end{description}
%
%
\small
~\vspace*{3cm}\\
\subsection*{Compl\'ements}
%
\begin{enumerate}
\item D\'emontrer la formule (\ref{eq:quadrature})
\item Pour chaque formule de quadrature 1-D et 2-D pr\'esent\'ee dans ce chapitre, d\'emontrer qu'elle est exacte pour les polyn\^omes jusqu'\`a un certain degr\'e.
\end{enumerate}

%
\normalsize


