\usepackage{beamerarticle}

\chapter{Approximation de problèmes mixtes}
\label{cha:appr-de-probl-1}

\section{Model Problems}
\label{sec:model-problems}

  We consider now model problems as systems of PDEs where several functions are
  unknowns and which don't play the same roles \emph{mathematically} and
  \emph{physically}.
  \begin{problem}{Stokes}
    \begin{equation}
      \label{eq:chmixte:98}
      \left\{\begin{array}[c]{rl}
          -\Delta u + \nabla p & = f\ \mbox{ in } \Omega\\
          \nabla \cdot u & = 0\ \mbox{ in } \Omega
        \end{array}\right.
    \end{equation}
    where $u: \Omega \mapsto \R{d}$ is a velocity and $p: \Omega \mapsto \R{}$
    is a pressure.
  \end{problem}
  \begin{problem}{Darcy}
    \begin{equation}
      \label{eq:chmixte:99}
      \left\{\begin{array}[c]{rl}
          \sigma + \nabla u & = f\ \mbox{ in } \Omega\\
          \nabla \cdot \sigma & = g\ \mbox{ in } \Omega
        \end{array}\right.
    \end{equation}
    where $\sigma: \Omega \mapsto \R{d}$ is a velocity and $u: \Omega \mapsto \R{}$
    is a hydraulic charge(pressure).

  \end{problem}


\section{Applications}

  We shall focus on Stokes, but the abstract setting of the next section is
  the same for Stokes and Darcy.

\begin{block}{Stokes and incompressible Navier-Stokes for Newtonian fluids}
  The Stokes model is the basis for fluid mechanics models and is a
  simplication of the Navier-Stokes equations where the viscous effects/terms
  are much bigger than the convective ones
  \begin{equation}
    \label{eq:chmixte:3}
    \left\{\begin{array}[c]{rl}
           \rho( \frac{\partial u}{\partial t} + u \cdot \nabla u) - \nu \Delta u + \nabla p & = f\ \mbox{ in } \Omega\\
           \nabla \cdot u & = 0\ \mbox{ in } \Omega
         \end{array}\right.
     \end{equation}
     The first equation results from the conservation of momentum and the second
     from the conservation of mass.
   \end{block}

  \begin{remark}{}
    The well-posedness of these problems results from a so-called \alert{inf-suf
    condition} which is not automatically transfered at the \emph{discrete level}.
  \end{remark}

  \begin{remark}{In practice}
    In order to ensure that the finite element approximation is well-posed, we
    will need to choose approximation spaces that satisfy a \emph{compatibility
    condition} that ensures that a \emph{discrete inf-sup condition} is satisfied.
  \end{remark}

\section{Saddle point problems}
\label{sec:saddle-point-probl}

\subsection{Abstract Continuous Setting}
  Denote
  \begin{itemize}
  \item $X$ and $M$ two Hilbert spaces\footnote{An euclidian space which is
  complete for the norm induced by the scalar product}
  \item two linear forms $f \in X'=\mathcal{L}(X, \R{})$ and $g \in
    M'=\mathcal{L}(M, \R{})$
  \item $a \in \mathcal{L}(X\times X, \R{})$ and $b \in \mathcal{L}(X\times M,
    \R{})$ two bilinear forms
  \end{itemize}
  We are interested in the following abstract problem:
  \begin{problem}[Abstract formulation]
    \label{prob:chmixte:1}
    Look for $(u,p) \in X \times M$ such that
    \begin{equation}
      \label{eq:chmixte:4}
      \left\{
        \begin{array}[c]{rl}
          a(u,v) + b(v,p) & = f(v), \quad \forall v \in X\\
          b(u,q) & = g(q), \quad \forall q \in M
        \end{array}
      \right.
    \end{equation}
  \end{problem}



\subsection{Definition of a saddle point problem}
  \begin{definition}[Saddle point problem]
    \label{def:chmixte:1}
    If the bilinear form $a$ is \emph{symmetric} and positive on $X\times X$, we
    say that problem~\ref{prob:chmixte:1} is a \emph{saddle point problem}.
  \end{definition}

  \begin{remark}{Structure of the problem}
    \begin{itemize}
    \item the space of solution is the same of the test space
    \item the unknown $p$ does not appear in the second equation
    \item the unknown functions $u$ and $p$ are coupled via the same bilinear
      form $b$ is the first and second equation.
    \end{itemize}
    The next question is : \alert{is the problem~\ref{prob:chmixte:1} well-posed?}
  \end{remark}

\section{Well posed problem}
\label{sec:well-posed-problemframe}


\subsection{Reformulation}
  Let's rewrite Problem~\ref{prob:chmixte:1}. Denote $V=X\times M$ and introduce $c
  \in \mathcal{L}(V\times V, \R{})$ such that
  \begin{equation}
    \label{eq:chmixte:5}
    c((u,p),(v,q)) = a(u,v)+b(v,p)+b(u,q)
  \end{equation}
  and $h\in \mathcal{L}(V,\R{})$ such that
  \begin{equation}
    \label{eq:chmixte:6}
    h(v,q) = f(v)+g(q)
  \end{equation}
  then problem~\ref{prob:chmixte:1} reads
  \begin{problem}
    \label{prob:chmixte:2}
    Look for $(u,p) \in V$ such that
    \begin{equation}
      \label{eq:chmixte:7}
        \begin{array}[c]{rl}
          c((u,p), (v,q)) & = h(v,q), \quad \forall (v,q) \in V
        \end{array}
      \end{equation}
  \end{problem}


  \begin{theorem}[Well posedness of the Saddle-Point problem]
    \label{thr:chmixte:1}
    We suppose that $a$ is coercive over $X$, the problem~\ref{prob:chmixte:2} is
    \emph{well-posed} if and only if the bilinear form $b$ satisfies the
    following inf-sup condition: there exists $\beta > 0$ such that
    \begin{equation}
      \label{eq:chmixte:8}
      \inf_{q \in M} \sup_{v \in X} \frac{b(v,q)}{||v||_X ||q||_M} \geq \beta
    \end{equation}
  \end{theorem}
  \begin{remark}
    \begin{itemize}
    \item Lax-Milgram provides only a sufficient condition for well-posedness
    \item the form $c$ in Problem~\ref{prob:chmixte:2} does not satisfy Lax-Milgram.
    \end{itemize}
  \end{remark}


  Let's introduce the so-called \emph{Lagrangian} $l \in \mathcal{L}(X\times M,
  \R{})$ defined by
  \begin{equation}
    \label{eq:chmixte:9}
    l(v,q) =  \frac{1}{2} a(v,v) + b(v,q) - f(v) - g(q)
  \end{equation}
  \begin{definition}[Saddle point]
    \label{def:chmixte:2}
    We say that the point $(u,p)\in X\times M$ is a saddle point of $l$ if
    \begin{equation}
      \label{eq:chmixte:10}
      \forall (v,q) \in X\times M, \quad l(u,q) \leq l(u,p) \leq l(v,p)
    \end{equation}
  \end{definition}
  \begin{proposition}[Unicity of the saddle point]
    \label{prop:chmixte:1}
    Under the hypothesys of theorem~\ref{thr:chmixte:1}, the Lagrangian $l$ defined
    by~\eqref{eq:chmixte:9} has a \emph{unique} saddle point. Moreover this saddle point
    is the the unique solution of problem~\ref{prob:chmixte:1}.
  \end{proposition}

\section{Finite element approximation}
\label{sec:finite-elem-appr}

\subsection{Abstract Discrete Problem}
  We now turn to the approximation of the problem~\ref{prob:chmixte:1} by a \emph{standard
  Galerkin method} in a \emph{conforming} way.

  Denote the two spaces $X_h \subset X$ and $M_h \subset M$, we consider the
  following problem:
  \begin{problem}
    \label{prob:chmixte:3}
    Look for $(u_h,p_h) \in X_h \times M_h$ such that
    \begin{equation}
      \label{eq:chmixte:11}
      \left\{
        \begin{array}[c]{rl}
          a(u_h,v_h) + b(v_h,p_h) & = f(v_h), \quad \forall v_h \in X_h\\
          b(u_h,q_h) & = g(q_h), \quad \forall q_h \in M_h
        \end{array}
      \right.
    \end{equation}
  \end{problem}


  \begin{theorem}[Well-posedness of the discrete problem]
    \label{thr:chmixte:2}
    We suppose that $a$ is coercive over $X$ and that $X_h \subset X$ and $M_h
    \subset M$. Then the problem~\ref{prob:chmixte:3} is \emph{well-posed} if and only
    if the following \emph{discrete inf-sup condition}  is satisfied: there
    exists $\beta_h  > 0$ such that
    \begin{equation}
      \label{eq:chmixte:12}
      \inf_{q_h \in M_h} \sup_{v_h \in X_h} \frac{b(v_h,q_h)}{||v_h||_{X_h} ||q_h||_{M_h}} \geq \beta_h
    \end{equation}
  \end{theorem}

  \begin{remark}{Compatibility condition}
    Problem~\ref{prob:chmixte:3} to be well posed, requires that the spaces $X_h$ and
    $M_h$ satisfy the condition~\eqref{eq:chmixte:12}. This is known as the
    Babuska-Brezzi (BB) or Ladyhenskaya-Babuska-Brezzi (LBB).
  \end{remark}


  Regarding error analysis, we have the following lemma
  \begin{lemma}
    \label{lem:1}
      Thanks to the Lemma of Céa applied to Saddle-Point Problems, the unique
      solution $(u,p)$ of problem~\ref{prob:chmixte:3}  satisfy
      \begin{equation}
        \label{eq:chmixte:13}
        \begin{array}[c]{rl}
          ||u-u_h||_X & \leq c_{1h} \inf_{v_h \in X_h}  ||u-v_h||_X + c_{2}
          \inf_{q_h \in M_h}  ||q-q_h||_M\\
          ||p-p_h||_X & \leq c_{3h} \inf_{v_h \in X_h}  ||u-v_h||_X + c_{4h} \inf_{q_h \in M_h}  ||q-q_h||_M
        \end{array}
      \end{equation}
      where
      \begin{itemize}
      \item $c_{1h} =
        (1+\frac{||a||_{X,X}}{\alpha})(1+\frac{||b||_{X,M}}{\beta_h})$ where
        $\alpha$ is the coercivity constant of $a$ over X.
      \item $c_{2} = \frac{||b||_{X,M}}{\alpha}$
      \item $c_{3h} = c_{1h} \frac{||a||_{X,X}}{\beta_h}$, $c_{4h} = 1+ \frac{||b||_{X,M}}{\beta_h}+\frac{||a||_{X,X}}{\beta_h}$
      \end{itemize}
  \end{lemma}
  \begin{remark}
    The constants $c_{1h}, c_{3h}, c_{4h}$ are as large as $\beta_h$ is small.
  \end{remark}



\subsection{Linear system associated}
\label{sec:linear-system}

The discretisation process leads to a linear system. We denote
  \begin{itemize}
  \item $N_u = \dim {X_h}$
  \item $N_p = \dim {M_h}$
  \item $\{\phi_i\}_{i=1,...,N_u}$ a basis of $X_h$
  \item $\{\psi_k\}_{k=1,...,N_p}$ a basis of $M_h$
  \item for all $u_h = \sum_{i=1}^{N_u} u_i \phi_i$, we associate $U \in
    \R{N_u}$, $U=(u_1,\ldots,u_{N_u})^T$, the component vector of $u_h$ is
    $\{\phi_i\}_{i=1,\ldots,N_u}$
  \item for all $p_h = \sum_{k=1}^{N_p} u_k \psi_k$, we associate $P \in
    \R{N_p}$, $P=(p_1,\ldots,p_{N_p})^T$, the component vector of $p_h$ is $\{\psi_k\}_{k=1,\ldots,N_p}$
  \end{itemize}

  \begin{block}{}
    The matricial form of problem ~\ref{prob:chmixte:3} reads
    \begin{equation}
      \label{eq:chmixte:15}
      \begin{bmatrix}
        \mathcal{A} & \mathcal{B}^T\\
        \mathcal{B} & 0
      \end{bmatrix}
      \begin{bmatrix}
        U \\
        P
      \end{bmatrix}
      =
      \begin{bmatrix}
        F\\
        G
      \end{bmatrix}
    \end{equation}
    where the matrix $\mathcal{A} \in \R{N_u,N_u}$ and $\mathcal{B} \in
    \R{N_p,N_u}$ have the coefficients
    \begin{equation}
      \label{eq:chmixte:16}
      \mathcal{A}_{ij} = a(\phi_j,\phi_i), \quad \mathcal{B}_{ki} = b(\phi_i,\psi_k)
    \end{equation}
    and the vectors $\mathcal{F} \in \R{N_u}$ and $\mathcal{G} \in \R{N_p}$
    have the coefficients
    \begin{itemize}
    \item $F_i=f(\phi_i)$
    \item $G_k=g(\psi_k)$
    \end{itemize}
  \end{block}


  \begin{remark}[Remarks on the linear system]
    \label{rem:2}
    \begin{enumerate}
    \item Since $a$ is symmetric and coercive, $\mathcal{A}$ is \emph{symmetric
       positive definite}
     \item The matrix of the system \eqref{eq:chmixte:15} is symmetric but not positive
     \item The inf-sup condition~\eqref{eq:chmixte:12} is equivalent to the fact that
       $\mathcal{B}$ is of \emph{maximum rank}, \emph{i.e.} $\ker(\mathcal{B}^T)
       = \{0 \}$.
     \item From theorem~\ref{thr:chmixte:2}, the matrix of the system~\eqref{eq:chmixte:15} is
       invertible
    \end{enumerate}
  \end{remark}


  \begin{block}{When the Inf-sup condition is not satisfied}
    The counter examples when the inf-sup condition~\eqref{eq:chmixte:12} is not
    satisfied(e.g. $\mathcal{B}$ is not maximum rank ) occur usually in two
    cases:
    \begin{enumerate}
    \item $\dim {M_h} > \dim {X_h}$: the space of pressure is too large for the
      matrix $\mathcal{B}$ to be maximum rank. In that case $\mathcal{B}$ is
      injective ($\ker(\mathcal{B}) = \{0\})$. We call this \emph{locking}.
    \item there exists a vector $Q^* \neq 0$ in $\ker(\mathcal{B}^T)$. The
      discrete field $q^*_h$ in $M_h$, $q^*_h=\sum_{k=1}^{N_p} Q^*_k \psi_k$,
      associated is called a \emph{spurious mode}. $q^*_H$ is such that
      \begin{equation}
        \label{eq:chmixte:14}
        b(v_h,q^*_h)=0.
      \end{equation}

    \end{enumerate}
  \end{block}

We now introduce the Uzawa matrix.
\begin{block}{Uzawa}
  The matrix
    \begin{equation}
      \label{eq:chmixte:17}
      \mathcal{U} = \mathcal{B} \mathcal{A}^{-1} \mathcal{B}^T
    \end{equation}
    is called the \emph{Uzawa matrix}. It is \emph{symmetric positive definite}
    from the properties of $\mathcal{A}$, $\mathcal{B}$
  \end{block}
  \begin{block}{Applications}
    The Uzawa matrix occurs when eliminating the velocity in
    system~\eqref{eq:chmixte:15} and get a linear system on $P$:
    \begin{equation}
      \label{eq:chmixte:18}
      \mathcal{U} P = \mathcal{B} \mathcal{A}^{-1} F - G
    \end{equation}
    then one application is to solve \eqref{eq:chmixte:15} by solving \eqref{eq:chmixte:18}
    iteratively and compute the velocity afterwards.
  \end{block}

\section{Mixed finite element for Stokes}
\label{sec:mixed-finite-element}

\subsection{Variational formulation }
\label{sec:vari-form-}

We start with the Well-posedness at the continuous level
  \begin{itemize}
  \item   We consider the model problem~\eqref{eq:chmixte:1} with homogeneous Dirichlet
  condition on velocity $u = 0$ on $\partial \Omega$
  \item   We suppose the $f \in [L^2(\Omega)]^d$ and $g \in L^2(\Omega)$ with
  \begin{equation}
    \label{eq:chmixte:20}
    \int_\Omega g = 0
  \end{equation}
    Introduce
  \begin{equation}
    \label{eq:chmixte:19}
    L^2_0(\Omega) = \Big\{ q \in L^2(\Omega): \int_\Omega q = 0 \Big\}
  \end{equation}

  \end{itemize}

  \begin{remark}{The condition~\eqref{eq:chmixte:20}}
    The condition~\eqref{eq:chmixte:20} comes from the divergence theorem applied to the
    divergence equation and the fact that $u=0$ on the boundary
    \begin{equation}
      \label{eq:chmixte:21}
      \int_\Omega g = \int_\Omega \nabla \cdot u = \int_{\partial \Omega} u
      \cdot n = 0
    \end{equation}
    This is a \emph{necessary} condition for the existence of a solution $(u,p)$
    for the Stokes equations with these boundary conditions.
  \end{remark}


We turn now to the variational formulation.  The Stokes problem reads
    \begin{problem}[Stokes Variational Formulation]
      \label{prob:chmixte:4}
      Look for $(u,p) \in [H^1_0(\Omega)]^d \times L^2_0(\Omega)$ such that
      \begin{equation}
        \label{eq:chmixte:25}
      \left\{
        \begin{array}[c]{rl}
          \int_\Omega \nabla u : \nabla v -\int_\Omega p \nabla \cdot v  & =
          \int_\Omega f \cdot v, \quad \forall v \in [H^1_0(\Omega)]^d\\
          \int_\Omega q \nabla \cdot u & = - \int_\Omega g q, \quad \forall q \in L^2_0(\Omega)
        \end{array}
      \right.
    \end{equation}
    \end{problem}
    We retrieve the problem~\ref{prob:chmixte:1} with $X=[H^1_0(\Omega)]^d$ and
    $M=L^2_0(\Omega)$ and
    \begin{equation}
      \label{eq:chmixte:22}
      \begin{array}[c]{rlrl}
      a(u,v) &= \int_\Omega \nabla u : \nabla v,& \quad b(v,p) &= -\int_\Omega p
      \nabla \cdot v,\\
      \quad f(v) &=  \int_\Omega f \cdot v,& \quad g(q) &= - \int_\Omega g q
      \end{array}
    \end{equation}
    \begin{remark}{Pressure up to a constant}
      The pressure is known up to a constant, that's why we look for them in
      $L^2_0(\Omega)$ to ensure uniqueness.
    \end{remark}



% \begin{frame}
%   \frametitle{Well-posedness}

%   \begin{lemma}
%     \label{lem:2}
%     Denote $\Omega$ a domain of $\R{d}, d \geq 2$, Then the operator
%     \begin{equation}
%       \label{eq:chmixte:23}
%       \nabla \cdot : [H^1_0(\Omega)]^d \mapsto L^2_0(\Omega)
%     \end{equation}
%     is surjective.
%   \end{lemma}

%   This lemma and the theorem
%

\subsection{Finite element approximation}
\label{sec:finite-elem-appr-1}

  Denote $X_h \subset [H^1_0(\Omega)]^d$ and $M_h \subset L^2_0(\Omega)$

  \begin{problem}[Discrete Stokes formulation]
    \label{prob:chmixte:5}
    Look for $(u_h,p_h) \in X_h \times M_h$ such that
    \begin{equation}
      \label{eq:chmixte:24}
      \left\{
        \begin{array}[c]{rl}
          \int_\Omega \nabla u_h : \nabla v_h + \int_\Omega p_h \nabla \cdot v_h
          & = \int_\Omega f \cdot v_h, \quad \forall v_h \in X_h\\
          \int_\Omega q_h \nabla \cdot u_h & = -\int_\Omega g q_h, \quad \forall q_h \in M_h
        \end{array}
      \right.
    \end{equation}
  \end{problem}
  \begin{remark}[Well-posedness]
    \label{rem:1}
    This problem, thanks to theorem~\ref{thr:chmixte:2} is well-posed if and only if
    $X_h$ and $M_h$ are such that there exists $\beta_h > 0$
    \begin{equation}
      \label{eq:chmixte:26}
      \inf_{q_h \in M_h} \sup_{v_h \in X_h} \frac{\int_\Omega q_h \nabla \cdot v_h}{||v_h||_{X_h} ||q_h||_{M_h}} \geq \beta_h
    \end{equation}
  \end{remark}

% \lstinputlisting{mystokes.cpp}

\subsection{Some counter examples: bad finite element for Stokes}
\label{sec:counter-examples}


In this section we present two classical bad finite element approximations.

\subsubsection{Finite element $\poly{P}_1/\poly{P}_0$: locking}
    thanks to the Euler relations


    \begin{equation}
      \label{eq:chmixte:28}
      \begin{array}[c]{rl}
        N_{\mathrm{cells}} - N_{\mathrm{edges}} + N_{vertices}  &= 1-I\\
      N^\partial_{\mathrm{vertices}} - N^\partial_{\mathrm{edges}} &= 0
      \end{array}
    \end{equation}
    where $I$ is the number of holes in $\Omega$.  We have that $\dim {M_h} =
    N_{\mathrm{cells}}$, $\dim {X_h} = 2 N^i_{\mathrm{vertices}}$ and so
    \begin{equation}
      \label{eq:chmixte:29}
      \dim {M_h} - \dim {X_h} = N_{\mathrm{cells}} - 2 N^i_{\mathrm{vertices}} =
      N^\partial_{\mathrm{edges}} - 2 > 0
    \end{equation}
    so $M_h$ is too rich for the condition \eqref{eq:chmixte:26} and we have
    $\ker(\mathcal{B}) = \{0\}$ such that the \emph{only} discrete $u_h^*$,
    with components $U^*$, satisfying $\mathcal{B} U^*$ is the nul field, $U^*=0$.


\subsubsection{Finite element $\poly{Q}_1/\poly{P}_0$: spurious mode}

    We can construct in that case a function $q_h^*$ on a uniform grid which is
    equal alternatively -1, +1 (chessboard) in the cells of  the mesh, then
    \begin{equation}
      \label{eq:chmixte:27}
      \forall v_h \in [Q^1_{c,h}]^d, \quad \int_\Omega q^*_h \nabla \cdot v_h = 0
    \end{equation}
    and thus the associated $X_h$, $M_h$ do not satisfy  the condition~\eqref{eq:chmixte:26}.

\subsubsection{Finite element $\poly{P}_1/\poly{P}_1$: spurious mode}
    We can construct in that case a function $q_h^*$ on a uniform grid which is
    equal alternatively -1, 0, +1 at the vertices of the mesh, then
    \begin{equation}
      \label{eq:chmixte:27}
      \forall v_h \in [P^1_{c,h}]^d, \quad \int_\Omega q^*_h \nabla \cdot v_h = 0
    \end{equation}
    and thus the associated $X_h$, $M_h$ do not satisfy  the condition~\eqref{eq:chmixte:26}.


\subsection{Mini-Element}
\label{sec:finite-elem-stok}

  The problem with the $\poly{P}_1/\poly{P}_1$ mixed finite element is that the
  velocity is not \emph{rich} enough. A cure is to add a function $v_h^*$ in the
  velocity approximation space to ensure that
  \begin{equation}
    \label{eq:chmixte:30}
    \int_\Omega q^*_h \nabla \cdot v_h^* \neq 0
  \end{equation}
  where $q_h^*$ is the spurious mode. To do that we add the bubble function to
  the $\poly{P}_1$ velocity space.
  \begin{definition}[Bubble function]
    \label{def:chmixte:3}
    Recall the construction of finite elements on a reference convex $\hat{K}$.
    We say that $\hat{b}: \hat{K} \mapsto \R{}$ is a bubble function if:
    \begin{itemize}
    \item $\hat{b} \in H^1_0(\hat{K})
$
    \item $0 \leq \hat{b}(\hat{x}) \leq 1, \quad \forall \hat{x} \in \hat{K}$
    \item $\hat{b}(\hat{C}) = 1, \quad \mbox{where} \hat{C}$ is the barycenter of $\hat{K}$
    \end{itemize}
  \end{definition}


  \begin{exampleblock}{Example}
    The function
    \begin{equation}
      \label{eq:chmixte:31}
      \hat{b} = (d+1)^{d+1} \Pi_{i=0}^d\ \hat{\lambda}_i
    \end{equation}
    where $(\hat{\lambda}_0, \ldots, \hat{\lambda}_d)$ denote the barycentric
    coordinates on $\hat{K}$
  \end{exampleblock}


  Denote $\hat{b}$ a bubble fonction on $\hat{K}$, we set
  \begin{equation}
    \label{eq:chmixte:32}
    \hat{P} = [\poly{P}_1(\hat{K}) \oplus \vect{} (\hat{b})]^d,
  \end{equation}
  and introduce
  \begin{eqnarray}
    \label{eq:chmixte:33}
    X_h &=& \Big\{ v_h \in [C^0(\bar{\Omega})]^d : \forall K \in \mathcal{T}_h, v_h
    \circ T_K \in \hat{P}; v_{h_|{\partial \Omega}} = 0 \Big\}\\
    M_h &=& P^1_{c,h}
  \end{eqnarray}

  \begin{lemma}[Compatibility condition]
    \label{lem:3}
    The spaces $X_h$ and $M_h \cap L^2_0(\Omega)$ satisfy the compatibility
    condition~\eqref{eq:chmixte:26} uniformly in $h$.
  \end{lemma}

  \begin{theorem}[Error estimation for the mini-element]
    \label{thr:chmixte:3}
    Suppose that $(u,p)$, solution of problem~\ref{prob:chmixte:1}, is smooth enough,
    ie. $u \in [H^2(\Omega)]^d \cap [H^1_0(\Omega)]^d$ and $p\in H^1(\Omega)
    \cap L^2_0(\Omega)$. Then there exists a constant $c$ such that for all $h
    >0$
    \begin{equation}
      \label{eq:chmixte:34}
      \| u- u_h \|_{1,\Omega} + \|p-p_h\|_{0,\Omega} \leq c h (\|u\|_{2,\Omega}
      +\|p\|_{1,\Omega})
    \end{equation}
    and if the Stokes problem is stabilizing then
    \begin{equation}
      \label{eq:chmixte:35}
      \|u-u_h\|_{0,\Omega} \leq c h^2 ( \|u\|_{2,\Omega} +\|p\|_{1,\Omega}).
    \end{equation}
  \end{theorem}
  \begin{definition}[Stabilizing Stokes problem]
    \label{def:chmixte:4}
    We say that the Stokes problem is stabilizing if there exists a constant
    $c_S$ such that for all $f \in [L^2(\Omega)]^d$, the unique solution $(u,p)$
    of \eqref{eq:chmixte:25} with $g=0$ is such that:
    \begin{equation}
      \label{eq:chmixte:36}
      \|u\|_{2,\Omega} + \|p\|_{1,\Omega} \leq c_S \|f\|_{0,\Omega}
    \end{equation}
    A sufficient condition for stabilizing Stokes problem is that the $\Omega$
    is a polygonal convex in 2D or of class $C^1$  in $\R{d}, d=2,3$.
  \end{definition}

\subsection{Taylor-Hood Element}
\label{sec:taylor-hood-element}


  The mini-element solved the compatibility condition problem, but the error
  estimation in equation~\eqref{eq:chmixte:35} is not optimal in the sense that
  \begin{enumerate}
  \item the pressure space is sufficiently rich to enable a $h^2$ convergence in
    the pressure error,
  \item but the velocity space is not rich enough to ensure a $h^2$ convergence
    in the velocity error.
  \end{enumerate}

  \begin{block}{Taylor-Hood element}
    The idea of the Taylor-Hood element is to enrich even more the velocity
    space to ensure optimal convergence in $h$. Here we will take $[\poly{P}_2]^d$ for
    the velocity and $\poly{P}_1$ for the pressure.
  \end{block}


  Introduce
  \begin{eqnarray}
    \label{eq:chmixte:39}
    X_h &=&  [P^2_{c,h}]^d\\
    M_h &=& P^1_{c,h}
  \end{eqnarray}

  \begin{lemma}[Compatibility condition]
    \label{lem:3}
    The spaces $X_h$ and $M_h \cap L^2_0(\Omega)$ satisfy the compatibility
    condition~\eqref{eq:chmixte:26} uniformly in $h$.
  \end{lemma}

  \begin{theorem}[Error estimation for the Taylor-Hood element]
    \label{thr:chmixte:3}
    Suppose that $(u,p)$, solution of problem~\ref{prob:chmixte:1}, is smooth enough,
    ie. $u \in [H^3(\Omega)]^d \cap [H^1_0(\Omega)]^d$ and $p\in H^2(\Omega)
    \cap L^2_0(\Omega)$. Then there exists a constant $c$ such that for all $h
    >0$
    \begin{equation}
      \label{eq:chmixte:40}
      \| u- u_h \|_{1,\Omega} + \|p-p_h\|_{0,\Omega} \leq c h^2 (\|u\|_{3,\Omega}
      +\|p\|_{2,\Omega})
    \end{equation}
    and if the Stokes problem is stabilizing then
    \begin{equation}
      \label{eq:chmixte:41}
      \|u-u_h\|_{0,\Omega} \leq c h^3 ( \|u\|_{3,\Omega} +\|p\|_{2,\Omega}).
    \end{equation}
  \end{theorem}


  \begin{block}{Generalized Taylor-Hood element}
    We consider the mixed finite elements $\poly{P}_k/\poly{P}_{k-1}$ and
    $\poly{Q}_k/\poly{Q}_{k-1}$ which allows to approximate the velocity and
    pressure respectively with, on Simplices
    \begin{eqnarray}
        \label{eq:chmixte:42}
        X_h &=&  [P^{k}_{c,h}]^d\\
        M_h &=& P^{k-1}_{c,h}
      \end{eqnarray}
      On Hypercubes
      \begin{eqnarray}
        \label{eq:chmixte:43}
        X_h &=&  [Q^{k}_{c,h}]^d\\
        M_h &=& Q^{k-1}_{c,h}
      \end{eqnarray}

      We then have
      \begin{equation}
      \label{eq:chmixte:40}
      \|u-u_h\|_{0,\Omega} + h ( \| u- u_h \|_{1,\Omega} + \|p-p_h\|_{0,\Omega} ) \leq c h^{k+1} (\|u\|_{k+1,\Omega}
      +\|p\|_{k,\Omega})
    \end{equation}
  \end{block}


\begin{block}{Other Discretization spaces}
  \begin{itemize}
   \item Discrete inf-sup condition: dictates the choice of spaces
   \item Inf-sup stables spaces:
	\begin{itemize}
	  \item $\mathbb Q_k$-$\mathbb Q_{k-2}$, $\mathbb Q_k$-$\mathbb Q^{disc}_{k-2}$
	  \item $\mathbb P_k$-$\mathbb P_{k-1}$, $\mathbb P_k$-$\mathbb P_{k-2}$, $\mathbb P_k$-$\mathbb P^{disc}_{k-2}$
	  \item Discrete inf-sup constant independent of $h$, but dependent on $k$
	\end{itemize}
  \end{itemize}
 \end{block}


\subsection{Numerical validation: Test case}

  We consider the Kovasznay solution of the steady Stokes equations, see Kovasznay \cite{kovasznay_test}. The exact solution is
\begin{equation}\label{kovasznay_problem}
\begin{array}{r c l}
 \displaystyle \mathbf{u}(x,y) & = & \displaystyle \prect{1 - e^{\lambda x } \cos (2 \pi y), \frac{\lambda}{2 \pi} e^{\lambda x } \sin (2 \pi y)}^T \\
 \displaystyle p(x,y) & = & \displaystyle -\frac{e^{2 \lambda x}}{2} \\
 \displaystyle \lambda & = & \displaystyle \frac{1}{2 \nu} - \sqrt{\frac{1}{4\nu^2} + 4\pi^2}.
\end{array}
\end{equation}
The domain is defined as  $\domain = (-0.5,1) \times (-0.5,1.5)$ and $\nu = 0.035$. The forcing term for the momentum equation is obtained from the solution and is
\begin{equation}
 \mathbf{f} = \left( e^{\lambda x}  \left( \left( \lambda^2 - 4\pi^2 \right) \nu \cos (2\pi y) - \lambda e^{\lambda x} \right), e^{\lambda x} \nu \sin (2 \pi y) (-\lambda^2 + 4 \pi^2)           \right)^T
\end{equation}
Dirichlet boundary conditions are derived from the exact solution.






%%% Local Variables:
%%% coding: utf-8
%%% mode: latex
%%% TeX-PDF-mode: t
%%% TeX-parse-self: t
%%% TeX-auto-save: t
%%% x-symbol-8bits: nil
%%% TeX-auto-regexp-list: TeX-auto-full-regexp-list
%%% TeX-master: "mef-intro"
%%% ispell-local-dictionary: "american"
%%% End:
